\section{University Process}
USA provides different types of student visas, each designed for specific educational purposes. The most common is the F-1 visa, which is intended for students enrolled in academic programs at accredited U.S. colleges, universities, or language training programs. The J-1 visa is for students participating in exchange programs, including those sponsored by educational or nonprofit institutions. The M-1 visa is designated for students attending vocational or non-academic institutions. Each visa type has its own eligibility criteria, application process, and regulations regarding employment and duration of stay.




Pursuing higher education (F-1 visa) in the United States requires careful planning and documentation. To successfully apply to a university, obtain I-20, secure a student visa requires detailed steps and comply with U.S. immigration laws.
To receive an offer letter from the U.S. universities , students must submit the following general documents, but some universities may have specific requirements:

\begin{itemize}[label=--]
    \item Academic transcripts (translated into English if necessary)
    \item Standardized test scores (SAT/ACT for undergraduate, GRE/GMAT for graduate)
    \item English proficiency test scores (TOEFL: 80+ or IELTS: 6.5+)
    \item Statement of Purpose (SOP)
    \item 2-3 Letters of Recommendation
    \item Resume/CV (primarily for graduate applicants)
    \item Copy of passport bio page
    \item Proof of finances (most universities require this during application)
\end{itemize}

\subsection{I-20 Issuance}
Once admitted, students must provide additional documents for the university to issue the Form I-20:


\subsubsection{F-1 Visa Requirements}
\begin{itemize}[label=--]
    \item Full-time enrollment: 12+ credits (undergraduate), 9+ credits (graduate)
    \item On-campus work allowed up to 20 hours/week during semesters
    \item Optional Practical Training (OPT): 12 months post-graduation; STEM fields eligible for a 24-month extension
    \item SEVIS compliance: must report changes in address, program, or institution
\end{itemize}

\subsection{Travel Regulations}
\begin{itemize}[label=--]
    \item Must carry a valid F-1 visa and a signed I-20 for re-entry to the U.S.
    \item Students may enter the U.S. up to 30 days before their program start date
\end{itemize}

\subsection{F-1 Visa Required Documents}
During U.S. embassy visa interview, carry the following documents:

\begin{enumerate}
    \item Valid passport (minimum 6 months validity beyond intended stay)
    \item Original I-20 form, signed by both institution and applicant
    \item DS-160 confirmation page (with barcode)
    \item SEVIS fee receipt (\$350)
    \item Visa appointment confirmation page
    \item Financial documents (must match I-20 estimates)
    \item Academic records (transcripts and certificates)
    \item University admission letter
    \item Proof of strong ties to Nepal (e.g., property documents, family affidavits)
\end{enumerate}

\subsection{Visa Application Steps}
To apply for an F-1 visa, follow these sequential steps:

\begin{enumerate}
    \item Complete the DS-160 form at \href{https://ceac.state.gov}{CEAC}
    \item Pay the MRV visa fee (\$185) through any Commercial Bank (Category 'A')
    \item Schedule visa interview via \href{https://www.ustraveldocs.com/np}{us travel docs}
    \item Pay the SEVIS fee (\$350) online at \href{https://www.fmjfee.com/}{I-901}
    \item Attend interview at the U.S. Embassy in Kathmandu
    \begin{itemize}[label=--]
        \item Common interview questions include:
        \begin{itemize}
            \item Why did student choose this university?
            \item How will student finance their education?
            \item What are the student plans after graduation?
        \end{itemize}
    \end{itemize}
    \item Visa Decision:
    \begin{itemize}
        \item \textbf{Approved}: Passport with visa returned in 3-–5 working days
        \item \textbf{Denied}: Students will receive a 214(b) refusal notice; reapplication is possible, but must address the reasons for denial. Students can use same SEVIS upto 3 times.
    \end{itemize}
\end{enumerate}

\subsubsection{Requirements for Study in USA}
\textbf{Bachelors:}
\begin{itemize}
    \item Completion of 12 years of schooling (10+2 or equivalent)
    \item Academic transcripts and certificates
    \item English proficiency (TOEFL/IELTS/PTE)
    \item SAT/ACT (for some universities)
    \item Statement of Purpose (SOP)
    \item Letters of Recommendation
    \item Financial documents
\end{itemize}
\textbf{Masters:}
\begin{itemize}
    \item Bachelor’s degree (16 years of education)
    \item Academic transcripts and certificates
    \item English proficiency (TOEFL/IELTS/PTE)
    \item GRE/GMAT (for some programs)
    \item Statement of Purpose (SOP)
    \item Letters of Recommendation
    \item Resume/CV
    \item Financial documents
\end{itemize}

\subsubsection{Documentation}
\begin{itemize}
    \item Academic certificates and transcripts
    \item Passport (valid for at least 6 months beyond intended stay)
    \item English proficiency test scores
    \item Standardized test scores (if required)
    \item SOP and Letters of Recommendation
    \item Financial documents (bank statements, affidavits)
    \item Application forms and fees
\end{itemize}


\textbf{Academic Requirements}

\textbf{Bachelor’s Degree (Undergraduate)}
\begin{itemize}
    \item \textbf{Minimum Qualifications:} Completion of SLC/SEE and +2 (higher secondary education).
    \item \textbf{Grade Requirement:} At least 60\% or a GPA of 2.6 on a 4.0 scale.
    \item \textbf{Gap Year:} Acceptable up to 2 years.
    \item \textbf{English Proficiency:}
    \begin{itemize}
        \item IELTS: Minimum overall score of 6.0, with no band below 5.5.
        \item TOEFL: Minimum score of 70.
        \item PTE: Minimum score of 50.
    \end{itemize}
    \item \textbf{Standardized Tests:} SAT or ACT scores may be required, depending on the university.
\end{itemize}

\textbf{Master’s Degree (Postgraduate)}
\begin{itemize}
    \item \textbf{Minimum Qualifications:} Completion of a relevant bachelor’s degree.
    \item \textbf{Grade Requirement:} At least 60\% or a GPA of 2.6 on a 4.0 scale.
    \item \textbf{Gap Year:} Acceptable up to 5 years.
    \item \textbf{English Proficiency:}
    \begin{itemize}
        \item IELTS: Minimum overall score of 6.5, with no band below 6.0.
        \item TOEFL: Minimum score of 80.
        \item PTE: Minimum score of 58.
    \end{itemize}
    \item \textbf{Standardized Tests:} GRE or GMAT scores may be required, depending on the program.
\end{itemize}

\textbf{Language Proficiency}

English is the primary medium of instruction in US universities, so Nepalese students should take an English proficiency test for study in the USA. Most universities accept scores from the following tests:

\begin{itemize}
    \item \textbf{IELTS (International English Language Testing System):}
    \begin{itemize}
        \item Undergraduate: Minimum overall band score of 6.0 to 6.5.
        \item Graduate: Minimum overall band score of 6.5 to 7.0.
    \end{itemize}
    \item \textbf{TOEFL iBT (Test of English as a Foreign Language - Internet-Based Test):}
    \begin{itemize}
        \item Undergraduate: Minimum score of 70 to 90.
        \item Graduate: Minimum score of 90 to 100.
    \end{itemize}
    \item \textbf{PTE Academic (Pearson Test of English):}
    \begin{itemize}
        \item Undergraduate: Minimum score of 50--58.
        \item Graduate: Minimum score of 58--65.
    \end{itemize}
    \item \textbf{Duolingo English Test (DET):}
    \begin{itemize}
        \item Undergraduate: Minimum score of 95--105.
        \item Graduate: Minimum score of 105--115.
    \end{itemize}
\end{itemize}

\textit{Note: While some universities may accept a ``Medium of Instruction'' letter from previous institution as proof of English proficiency, standardized test scores are generally preferred and often mandatory. Always verify the specific requirements of each university.}

\textbf{Financial Requirements for Nepalese Students}
Financial requirements for Nepalese students applying to universities in the USA vary based on the level of education (Bachelor's or Master's) and the type of institution (public or private). The estimated costs include tuition, living expenses, and additional fees.
\textbf{Bachelor's Degree:}
\begin{itemize}
    \item Public Universities: Approximately \$20,000 -- \$40,000 per year for tuition.
    \item Private Universities: Approximately \$30,000 -- \$60,000 per year for tuition.
    \item Community Colleges (first two years): Approximately \$6,000 -- \$20,000 per year for tuition (a more affordable option).
    \item Total Estimated Annual Cost (including living expenses): \$35,000 -- \$60,000/year
    \item Graduate Programs: Annual tuition varies between \$30,000 and \$90,000, influenced by the university's prestige and the chosen field of study.
\end{itemize}

\textbf{Master's Degree:}
\begin{itemize}
    \item Public Universities: Approximately \$20,000 -- \$45,000 per year for tuition.
    \item Private Universities: Approximately \$22,000 -- \$60,000 per year for tuition.
    \item Total Estimated Annual Cost (including living expenses): \$35,000 -- \$65,000/year. (Some sources suggest a minimum bank balance of \$30,000 to \$50,000 for Master's programs).
\end{itemize}

\textbf{Living Expenses}
\begin{itemize}
    \item Accommodation: Typically, between \$8,000 and \$15,000 per year.
    \item Meals and Groceries: Approximately \$3,000 to \$6,000 annually.
    \item Health Insurance: Ranges from \$500 to \$3,000 per year.
    \item Transportation and Personal Expenses: Around \$1,000 to \$3,000 for transportation and \$2,000 to \$4,000 for personal expenses.
\end{itemize}

\textbf{Additional Costs}
\begin{itemize}
    \item Books and Supplies: Estimated at \$1,000 to \$2,000 annually.
    \item Visa and SEVIS Fees: The F-1 visa application fee is approximately \$160, with an additional \$350 SEVIS fee.
    \item Travel Expenses: Airfare and other travel-related costs can amount to \$1,000 to \$2,000.
\end{itemize}



