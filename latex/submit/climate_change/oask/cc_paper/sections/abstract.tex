Nepal's diverse climate, spanning from subtropical lowlands to arctic high mountains, poses significant challenges for its predominantly agriculture-dependent population. There is a gap regarding the impact of climate change on yield. This study seeks to address the knowledge gaps and examines the impact of climate variability on agricultural productivity and food security in Banke, Nepal, over the period 1990--2020. Analysis of long-term climatic data revealed a minor but steady increase in average temperature (0.0946\,°C per year), a significant upward trend in sunshine hours (15.15 hours per year, $R^2 = 0.3125$), and a modest annual rise in accumulated rainfall (approximately 1.94\,mm per year). Correlation analysis indicates that sunshine hours have a significant positive effect on crop yield ($r = 0.417$, $p = 0.017$), whereas rainfall and temperature exhibit weaker, statistically non-significant relationships. A linear regression model incorporating these variables explained 24.2\% of the variation in crop yield. Additionally, the region's agricultural profile is characterized by a high reliance on monoculture, limited crop diversification, and minimal institutional support despite generally sufficient food production. Shifts in cropping calendars and irrigation practices—most notably, the higher yields achieved under year-round irrigation compared to rainfed systems—further underscore farmers' adaptive responses to climate uncertainties. These findings highlight the dominant role of solar radiation in driving crop productivity and underscore the need for integrated, climate-resilient strategies to ensure long-term agricultural sustainability and food security. There is a pressing need for integrated, climate-resilient approaches that encompass better irrigation systems, diverse cropping patterns, and stronger institutional support.


\textbf{Keywords:} Climate variability, Agricultural productivity, Food security, Irrigation and Crop yield, Climate change adaptation

