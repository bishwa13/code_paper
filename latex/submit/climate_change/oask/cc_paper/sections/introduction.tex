Nepal’s climatic diversity, spanning from subtropical to arctic zones in its high mountains, is a unique geographical feature. Given that the majority of the population relies on agriculture for their livelihood, understanding the factors that influence agricultural output is crucial. Nepal's long-term agricultural development is increasingly at risk due to climate change events, such as erratic rainfall, droughts, and floods, which significantly affect crop production, food security, and livelihoods. Additionally, the agriculture sector is vulnerable to pests and diseases, which are further exacerbated by climate change \citep{gyawaliOverviewAgricultureNepal2021}.

\citet{mallaClimateChangeIts2009} highlight the multifaceted impacts of climate change on Nepalese agriculture, emphasizing the sensitivity of various crops to alterations in temperature, rainfall, and humidity, which influence pest and disease dynamics and potentially harm crop yields, with maize, for instance, showing a differential response to temperature increases, yielding more favorably in the mountainous regions than in the Terai and hills.
 
Climate change significantly impacts agriculture by altering key climatic factors such as temperature, carbon dioxide (CO$_2$) levels, and precipitation patterns. While moderate increases in temperature and CO$_2$ may enhance crop productivity in certain regions, extreme climate variations—such as more frequent droughts and floods—pose serious challenges to farmers. These climatic disruptions exacerbate existing stressors, including population growth and resource scarcity, amplifying their adverse effects on agricultural productivity and food security \citep{global_paudel_2015}. \citet{regmiCROPYIELDRESPONSE2019} examines the effect of climate change on agriculture, which manifests in various forms, including changes in average temperature, rainfall, and climate extremes, and analyzes crop yield responses to these climatic changes.

Food security is ensured when everyone has consistent access to sufficient food for an active and healthy life \citep{mulunehImpactClimateChange2021}. According to \citet{kangClimateChangeImpacts2009} Food security depends on four key factors: food availability, stability, access, and utilization. Increased climate variability, along with more frequent and intense weather events, puts significant pressure on food stability. Additionally, climate change impacts food quality by raising temperatures and shortening crop growth periods. Food security is impacted by climate change, especially in areas and populations where rain-fed agriculture is the primary source of food. Plants and crops have thresholds that when exceeded impair yield and growth \citep{mulunehImpactClimateChange2021}. Climate change is expected to intensify the frequency and severity of extreme weather events, leading to increased climate variability and uncertainty. Farmers in low-income countries are particularly vulnerable to these changes due to their high exposure to climate risks and limited adaptive capacity.  As climate extremes become more frequent, the ability of farmers in these regions to sustain agricultural productivity and food security will be increasingly challenged \citep{budhathokiAssessingFarmersPreparedness2020}. 

One of the most significant issues in the 21st century is to supply enough food for the growing population while maintaining the already strained ecosystem, which is endangered by climate change \citep{kangClimateChangeImpacts2009}. With the global population steadily increasing, the need to enhance production to meet the growing demand for food has become crucial. However, this imperative has prompted a quest for sustainable agricultural approaches that not only boost productivity but also prioritize the preservation of resources for the benefit of future generation \citep{singhSocioeconomicStatusQualitative2022}. 

The effects of climate change on agricultural output frequently interact with those on water availability productivity and soil water balance. Climate change affects temperature and precipitation, which immediately affect the state of soil moisture and groundwater levels. Crop types, planting locations, soil degradation, the growing environment, and the availability of water throughout the crop growth period all have an impact on crop production \citep{risalImpactClimateChange2022}.

Nepal's geographical diversity, ranging from the Terai plains to the Himalayan peaks, makes it highly vulnerable to climate change. Agriculture is a critical sector in Nepal, forming the backbone of the economy and providing livelihoods for nearly two-thirds of the population. However, it faces numerous challenges, including poverty, limited access to resources, and climate change. Climate change significantly impacts agriculture by altering temperature and precipitation patterns, leading to shifts in soil moisture and groundwater levels \citep{gyawaliOverviewAgricultureNepal2021}.

In particular, the Western Terai region, known as the breadbasket of Nepal, has experienced observable shifts in climate patterns, including an annual temperature rise of approximately 0.040°C. Erratic weather events and climate variability significantly threaten agriculture, which forms the backbone of the region's economy \citep{factors_dahal_2021}.  In order to improve agricultural resilience, production, food security, and sustainability, adaptive research should concentrate on creating crop varieties that can withstand drought, heat, and floods, including locally grown, indigenous, disease- and pest--resistant cultivars. It should also invest in resource centers, adjust sowing timing based on rainfall, encourage early maturing cultivars, ensure high-quality seeds and planting materials, and implement food production and self-reliance initiatives in rural, food-deficient areas \citep{khanalFactorsMotivatingFarmers2021}.

In the Banke district, challenges such as limited irrigation infrastructure and agricultural constraints have exacerbated food deficits. For instance, in 2015-16, Banke faced a food deficit of -1767 MT, highlighting the growing issue of food insecurity \citep{gyawaliOverviewAgricultureNepal2021}. Janaki Rural Municipality, situated in Banke, exemplifies the urgent need to understand the interplay between climate change impacts, agricultural resilience, and food security dynamics. 

This study seeks to address the knowledge gap regarding the impact of climate change on agriculture in Banke. By examining the relationship between climatic variables (temperature, precipitation, and sunshine hours) and crop yields over the past three decades, the research aims to provide a comprehensive understanding of the region's vulnerabilities. Additionally, the study explores farmers' perceptions of climate change, food security scenarios, and the current state of irrigation. It also investigates how these factors collectively influence agricultural resilience and food production.

The findings from this study can inform strategic planning and policy-making to enhance agricultural resilience and food security in Banke. By identifying the key challenges and opportunities, the research aims to support local and provincial governments in developing targeted interventions, such as promoting drought-resistant crops, improving irrigation systems, and providing training on adaptive farming practices. These measures are essential to ensuring sustainable agricultural development and improving the livelihoods of farmers in Janaki Rural Municipality, Banke.
