\subsection{Study Area}



The study area of this research is Janaki Rural Municipality of Nepalgunj, Banke (Figure ). The elevation of Janaki Rural Municipality is 165m above sea level. It is bordered by Nepalgunj Sub Metropolitan in the east, Kohalpur Municipality in the north, Khajura Rural Municipality in the west, and India in the south (Janaki Rural Municipality, 2075). The region experiences a temperate climate, characterized by flat terrain. According to the meteorological department records, the average annual rainfall is 1445.58 mm. The maximum recorded temperature reaches 46°C, while the minimum temperature is 4.5°C (Janaki Rural Municipality, 2075). Farming is the primary income source for most families. Of the 5,063 ha of cultivable land, only 2,532 ha is utilized, with 1,798 ha as farmland and 774 ha as upland.

In Janaki Rural Municipality, 1,294 households own less than 0.167 ha, 1,222 own 0.2–0.33 ha, 2,338 own 0.37–0.67 ha, 1,512 own 0.7–1 ha, and 973 own more than 20 ha, while 52 households are landless. Of the land, only 321 ha is irrigated year-round, and 1,871 ha lack irrigation. The municipality has 50 ponds, some used for irrigation. Despite abundant arable land, limited irrigation and technological support prevent commercial farming, leading many to rely on India for food supplies (Janaki Rural Municipality, 2075).

Janaki Rural Municipality consists of six wards in total, out of which three wards were randomly sampled and included in the research study as follows (Table \ref{tab:study_area}): 

\begin{table}[h]
\centering
\caption{Study Area with elevation and coordinates}
\label{tab:study_area}
\resizebox{\textwidth}{!}{
\begin{tabular}{|c|l|c|l|}
\hline
\textbf{S.N.} & \textbf{Study Location} & \textbf{Elevation (m)} & \textbf{Coordinates} \\ \hline
1 & Janaki Rural Municipality – 01, Saigaun & 164 & 28°02'42"N 81°34'05"E \\ \hline
2 & Janaki Rural Municipality – 03, Indrapur & 172 & 28°05'02"N 81°36'20"E \\ \hline
3 & Janaki Rural Municipality – 04, Khajura Khurda & 163 & 28°06'26"N 81°36'00"E \\ \hline
\end{tabular}
}
\end{table}

Source: (Janaki Rural Municipality, 2075)

\subsection{Methods of Data Collection}

\subsubsection{Primary Source of Data}
A comprehensive household survey and intensive field research were carried out in Janaki Rural Municipality's wards 01, 03, and 04, which included a total of 47 villages. Structured interviews and direct observations were used to collect data on agriculture, irrigation, food security, and perceptions of climate change. Village sample sizes were assessed using household ratios, and then systematic sampling was conducted.

\subsubsection{Secondary Source of Data}
The secondary data were taken from literature reviews relevant to the research topic, where the majority of reviews were collected from desk studies, i.e., the internet. Various knowledge and data were achieved through reference books, recently published national newspapers, international journals, reports, records, past literature, reviews of various websites, and so on.

\subsubsection{Sampling Frame}
Stratified random sampling was utilized in a survey conducted in Janaki Rural Municipality, where wards 1, 3, and 4 were chosen by lottery out of six wards. Villages were sampled proportionally, whereas households within each village were systematically selected based on municipality-provided household data. Janaki Rural Municipality, divided into six wards, had wards 1, 3, and 4 selected for the study using a lottery method. Ward 1 includes 5 villages, while wards 3 and 4 have 21 villages each, as identified by an internal survey in 2075 B.S. Villages were proportionately sampled through stratified random sampling to ensure diversity and minimize sampling errors. Observations were also conducted to gather information on homes within each village.

The details of the household distribution are provided in Table \ref{tab:household_distribution}.

\begin{table}[h]
\centering
\caption{Household Distribution by Ward}
\label{tab:household_distribution}

\begin{tabular}{|c|c|c|}
\hline
\textbf{Ward} & \textbf{Household} & \textbf{Soil Sample} \\ \hline
1 & 1242 & 8 \\ \hline
3 & 1929 & 8 \\ \hline
4 & 1050 & 8 \\ \hline
\textbf{Total} & 7391 & 24 \\ \hline
\end{tabular}
\end{table}



\subsection{Sample Size Determination}
Total household population was obtained from the report obtained from Janaki Rural Municipality from the internal survey conducted in 2075 B.S. The sample size of the population to conduct the questionnaire survey was calculated at a 95\% confidence level using the formula:

\[
n = \frac{N \cdot z^2 \cdot p \cdot q}{(N - 1) \cdot e^2 + z^2 \cdot p \cdot q}
\]

\textbf{Source:} Dahal, 2021

Where:
\begin{itemize}
    \item $n$ = Sample size
    \item $N$ = Total number of households of selected wards
    \item $z$ = Confidence level at 95\%, $z = 1.96$
    \item $p$ = Expected rate of occurrence = 0.9
    \item $q$ = $1 - p = 0.1$ (expected rate of non-occurrence)
    \item $e$ = Degree of error = 0.05
\end{itemize}

\[
n = \frac{4221 \cdot (1.96)^2 \cdot 0.9 \cdot 0.1}{(4221 - 1) \cdot (0.05)^2 + (1.96)^2 \cdot 0.9 \cdot 0.1}
\]

\[
n = 133.9 \approx 134
\]

The selection of households from each ward was calculated using the following formula:

\[
n_h = \frac{N_h}{N} \cdot n
\]

\textbf{Source:} Dahal, 2021

Where:
\begin{itemize}
    \item $n_h$ = Sample size for each ward
    \item $N_h$ = Population size of each ward
    \item $N$ = Total population size = 4221
    \item $n$ = Total sample size = 134
\end{itemize}

\subsection{Sample Size of Selective Wards}

The sample size for each ward is calculated as follows:

\begin{table}[h]
\centering
\caption{Sample Size of Selective Wards}
\label{tab:sample_size_wards}
\begin{tabular}{|c|c|c|c|}
\hline
\textbf{Ward} & \textbf{$N_h$} & \textbf{$N_h / N$} & \textbf{$N_h / N \cdot n$} \\ \hline
1 & 1242 & 0.29 & 39 \\ \hline
3 & 1929 & 0.457 & 61 \\ \hline
4 & 1050 & 0.249 & 33 \\ \hline
\textbf{Total} & 4221 & & 134 \\ \hline
\end{tabular}
\end{table}

\subsection{Sample Size of Selective Villages}

The sample size for selective villages of ward 1 (Saigaun) is shown in Table \ref{tab:sample_size_village_ward1}:

\begin{table}[h]
\centering
\caption{Sample Size of Selective Village of Ward 1 (Saigaun)}
\label{tab:sample_size_village_ward1}
\resizebox{\textwidth}{!}{
\begin{tabular}{|c|c|c|c|}
\hline
\textbf{Village} & \textbf{Household Number} & \textbf{Sample Size} & \textbf{Sample Size Cumulated} \\ \hline
3 & 414 & 13 & 13 \\ \hline
5 & 331 & 10.39 & 10 \\ \hline
\textbf{Total} & 1242 & 39 & 39 \\ \hline
\end{tabular}
}
\end{table}

The sample size for selective villages of ward 3 (Indrapur) is shown in Table \ref{tab:sample_size_village_ward3}:

\begin{table}[h]
\centering
\caption{Sample Size of Selective Village of Ward 3 (Indrapur)}
\label{tab:sample_size_village_ward3}
\resizebox{\textwidth}{!}{
\begin{tabular}{|c|c|c|c|}
\hline
\textbf{Village} & \textbf{Household Number} & \textbf{Sample Size} & \textbf{Sample Size Cumulated} \\ \hline
1 & 127 & 4.02 & 4 \\ \hline
3 & 85 & 2.69 & 3 \\ \hline
19 & 149 & 4.71 & 5 \\ \hline
\textbf{Total} & 1929 & 61 & 61 \\ \hline
\end{tabular}
}
\end{table}

The sample size for selective villages of ward 4 (Khajura Khurda) is shown in Table \ref{tab:sample_size_village_ward4}:

\begin{table}[h]
\centering
\caption{Sample Size of Selective Village of Ward 4 (Khajura Khurda)}
\label{tab:sample_size_village_ward4}
\resizebox{\textwidth}{!}{
\begin{tabular}{|c|c|c|c|}
\hline
\textbf{Village} & \textbf{Household Number} & \textbf{Sample Size} & \textbf{Sample Size Cumulated} \\ \hline
3 & 50 & 1.57 & 2 \\ \hline
10 & 62 & 1.95 & 2 \\ \hline
15 & 87 & 2.73 & 3 \\ \hline
\textbf{Total} & 1050 & 33 & 33 \\ \hline
\end{tabular}
}
\end{table}

\subsubsection{Data Collection and Calculation}
The initial step was to collect the necessary data and the extractions of pertinent factors in order to accomplish the main goal of the examining the production trend and climatic scenario along with other variables and their correlation in the research area. Subsequently, these factors were scrutinized to identify their inter relationship and trends. The data used for this analysis were sourced from a variety of outlets, encompassing both primary and secondary data, along with ancillary data, to ensure a comprehensive and high-quality analysis. The process of gathering data involved accessing a rich pool of information from government reports, academic literature, and books, thereby offering a more expansive context and historical viewpoint for the research.
Moreover, adding information from trustworthy websites and internet sources improved the dataset's currency and depth. This diverse approach to data collection ensured the study's dependability and thoroughness while also enhancing its richness. Important contextual facts were acquired from reputable organizations including DHM, and MOALD. These details contributed to a thorough understanding of the issues influencing the research region. The primary, secondary, and auxiliary sources were all covered by this comprehensive data collection methodology. These many techniques for gathering data have enabled this thesis to get an accurate outcome to address the objectives.
IBM SPSS statistical 27 tool and Microsoft Excel was used to carry statistical analysis. 

\paragraph{Pearson Correlation Coefficient}
Pearson correlation coefficient is used for yield, sunshine hour, and accumulated rainfall to calculate the linear relationship between them before simple linear regression, where total yield is the dependent variable, and sunshine radiation, accumulated rainfall, and average temperature are the independent variables.

Pearson correlation coefficient (r) is a popular method for calculating a linear correlation. The degree and direction of the relationship between two variables are indicated by the correlation coefficient, which has a range of -1 to 1 Turney2022. 

A perfect positive correlation ($r = 1$) means that as one variable rises, the other rises in proportion.
A complete negative correlation ($r = -1$) means that as one variable rises, the other falls proportionately.
($r = 0$) indicates no linear correlation between the variables.

The formula for the Pearson correlation coefficient (r) between two variables (X) and (Y) with (n) data points is given by:
\[
r = \frac{n \sum_{i=1}^{n} (x_i - \bar{x})(y_i - \bar{y})}{\sqrt{\sum_{i=1}^{n} (x_i - \bar{x})^2} \sqrt{\sum_{i=1}^{n} (y_i - \bar{y})^2}}
\]

Where:
\begin{itemize}
    \item $X_i$ and $Y_i$ are individual data points for variables X and Y, respectively.
    \item $\bar{X}$ and $\bar{Y}$ are the means (average) of the variables X and Y, respectively.
    \item $\sum$ represents the summation symbol, indicating that you need to sum the quantities inside it across all data points.
\end{itemize}

Source: Turney2022

\paragraph{Simple Linear Regression Analysis}
Simple linear regression analysis is used in this study to establish the relationship between the dependent variable (Yield) and independent variables (climate data). This statistical technique allows for the examination of how changes in one variable (independent variable) affect changes in another variable (dependent variable) through the estimation of a linear relationship between them Mali2023.

The equation of simple linear regression is:
\[
Y_i = \beta_0 + \beta_1 X_i + \epsilon
\]

Where:
\begin{itemize}
    \item $Y_i$ represents the dependent variable; $X_i$ represents the independent variable.
    \item The intercept, denoted by $\beta_0$, indicates the value of $Y$ when $X$ equals 0.
    \item $\beta_1$ is the slope of the line, representing the change in $Y$ for a unit change in $X$.
    \item $\epsilon$ represents the error term, accounting for the differences between the observed and predicted values.
\end{itemize}

\paragraph{T-test}
The T-test is used to compare the average yields (in kilograms per hectare) of two groups categorized by different irrigation methods: year-round irrigation and rainfed irrigation during the monsoon season.

\begin{itemize}
    \item \textbf{Year-round Irrigation:} Agricultural plots where irrigation is consistently applied throughout the year.
    \item \textbf{Rainfed Irrigation:} Agricultural plots relying solely on rainfall during the monsoon season for irrigation.
\end{itemize}

Statistical software package (SPSS) is used for analysis. The t-test is performed to compare the average yields between the two irrigation groups. The significance level ($\alpha$) is set a priori to 0.05.

\[
H_0: \text{There is no difference in yields between the two groups.}
\]
\[
H_1: \text{There is a difference in yields between the two groups.}
\]

A statistical test for comparing the means of two groups is called a t-test. It is frequently employed in hypothesis testing to ascertain whether two groups are distinct from one another or whether a procedure or treatment genuinely affects the population of interest Bevans2020.

The t-test statistic is given by the formula:
\[
t = \frac{\overline{x_1} - \overline{x_2}}{\sqrt{S^2 \left( \frac{1}{n_1} + \frac{1}{n_2} \right)}}
\]

Where:
\begin{itemize}
    \item $\overline{x_1}$ and $\overline{x_2}$ represent the means of the two samples.
    \item $S^2$ represents the pooled variance of the two samples.
    \item $\frac{1}{n_1} + \frac{1}{n_2}$ represents the sizes of the two samples.
\end{itemize}

Source: Student1908, Zabell2008

\paragraph{Index Model}
An index model determines the index value for every unit area and uses those values to create a ranking map. An index model and a binary model are comparable in that they both rely on overlay operations for data processing and require multicriteria assessment. However, an index model generates an index value, not just a yes or no, for every unit area \cite{OneStopGIS2020}.

Two tiers of evaluation are applied to certain variables:
\begin{itemize}
    \item Considering relative significance and weighing.
    \item Values that are seen are assessed and assigned scores OneStopGIS2020.
\end{itemize}

\paragraph{Climate Change and Production Trend Analysis}

The simple linear regression analysis was carried out using IBM SPSS Statistical 27 tool to identify the relationship between the dependent and independent variables. In this data, production yield data is the dependent variable, and climatic variables such as accumulated precipitation data, average temperature, and sunshine radiation are the independent variables.

Initially, the correlation between these variables was computed, and afterward, the linear relationship between climate variables (sunshine hours, temperature, and rainfall) and yield was analyzed through simple linear regression analysis.

\paragraph{Climate Change Trend}

Three variables for climate change were studied, which included Accumulated Rainfall, Average Temperature, and Sunshine hours of the study area for the last thirty years, from 1990 to 2021. The data were obtained from the Department of Hydrology and Meteorology. Excel was used for plotting the linear trend graph.

\paragraph{Climate Change Perception}

An index model has been used to evaluate the perception of farmers towards climate change and its impact on agricultural production. Understanding of climate change and its impacts is deeply rooted in the perception of farmers towards climate change. A total of 134 respondents were asked to give their opinion on the following statement.

A three-point scale was used with the frequency of agreement/approval as +1, frequency of disagreement/disapproval as -1, and "don’t know/absent" was given as 0.

The following formula by Dahal (2021) has been used for calculating the index for binary variables:

\[
\text{INDEX} = \frac{FA(+1) + FDA(-1) + FDK(0)}{N}
\]

Where:
\begin{itemize}
    \item $FA$: Frequency of agreement/approval
    \item $FDA$: Frequency of disagreement/disapproval
    \item $FDK$: Frequency of "don’t know/absent"
    \item $N$: Total number of respondents = 134
\end{itemize}

For variables with multiple categories, the following formula has been used to calculate the index based on a data normalization technique (Nguyen 2019):

\[
\text{INDEX} = \frac{(\text{Max value} - \text{Min value}) \times X}{(\text{Max value} + \text{Min value}) \times \text{Max value}}
\]

Where $X$ is the value of a particular category.