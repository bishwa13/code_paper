

This study examined the interplay between climate variability and agricultural productivity in Banke and Janaki Rural Municipality over the period 1990–2020. Our analysis revealed a minor but steady increase in average temperature (0.0946°C per year), a significant upward trend in sunshine hours (15.15 hours per year, R² = 0.3125), and a modest annual increase in accumulated rainfall (approximately 1.94 mm per year). Notably, sunshine hours were found to have a significant positive correlation with crop yield (r = 0.417, p = 0.017), while rainfall and temperature exhibited weaker and statistically non-significant relationships with yield. The linear regression model, incorporating these climate variables, explained 24.2\% of the variation in crop yield, indicating moderate explanatory power.

In parallel, the agricultural profile of the region is characterized by a high reliance on monoculture, with limited crop diversification and minimal institutional support, despite overall sufficient food production. Shifts in crop calendars, particularly the delayed sowing and harvesting of major crops such as paddy, wheat, mustard, lentils, and pigeon pea, reflect farmers' adaptive responses to irregular monsoons and broader climatic shifts. Moreover, the irrigation analysis underscores that while 93.3\% of the land depends on rainfed irrigation, year-round irrigation—though practiced on a limited scale—significantly enhances crop yield (7868.63 kg/ha compared to 5450.43 kg/ha under rainfed conditions, t(132) = 2.696, p = 0.008).

Collectively, these findings underscore the dominant role of solar radiation in driving crop productivity and highlight the complex challenges posed by climate variability. They also reveal critical vulnerabilities in the current agricultural practices, particularly the dependence on rainfed irrigation and monoculture systems, which may exacerbate the impacts of climate-induced stresses. To ensure long-term agricultural sustainability and food security, there is an urgent need for integrated climate-resilient strategies that include improved irrigation infrastructure, diversified cropping systems, and enhanced institutional support.
