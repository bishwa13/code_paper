\subsection{Study Area}
Koshi Basin is situated in the eastern part of Nepal and constitutes one of the five major river basins in the country, recognized as the largest. This transboundary river, known as the Koshi River, meanders through China, Nepal, and India, with its origin in China before flowing southward into Nepal and on\% in China, 45\% in Nepal, and 23\% in India \parencite{bastakoti_agriculture_2017}. The study area was delineated by creating a basin map, with Chatara assumed as the river's outlet point. The elevation ranges from 65 m MSL in the Terai to over 8848 m MSL in the High Himalaya.


\begin{figure}[H]  
  \centering
  \includegraphics[width=0.8\textwidth]{images/study_area.png} 
  \caption{Selected stations for the study in the Koshi basin of Nepal}  
  \label{fig:study_area} 
\end{figure}


\begin{table}[H]
  \centering
  \caption{Descriptions of the selected stations included in this study}
  \label{tab:station_descriptions}
  \begin{tabularx}{\textwidth}{|X|X|X|X|X|X|}
      \toprule % Replaces the first \hline
      \textbf{Station No.} & \textbf{Station Name} & \textbf{Latitude (°N)} & \textbf{Longitude (°E)} & \textbf{Elevation (m)} & \textbf{Physiographic Regions} \\
      \midrule % Replaces \hline between header and body
      1316 & Chatara           & 26.82044 & 87.15917 & 105  & TAR \\
      1201 & Namche Bazar      & 27.81667 & 86.71667 & 3450 & H \\
      1401 & Olangchuhg G      & 27.68333 & 87.78333 & 3119 & H \\
      1225 & Syangboche        & 27.81667 & 86.71667 & 3700 & H \\
      1218 & Tengboche         & 27.83333 & 86.76667 & 3857 & H \\
      1206 & Okhaldhunga       & 27.30812 & 86.50423 & 1731 & MM \\
      1405 & Taplejung         & 27.35861 & 87.67    & 1744 & MM \\
      1103 & Jiri              & 27.63045 & 86.23211 & 1877 & MM \\
      1036 & Panchkhal         & 27.64513 & 85.62088 & 857  & MM \\
      1016 & Sarmathang        & 27.94456 & 85.59514 & 2574 & HM \\
      1123 & Manthali          & 27.3947  & 86.06123 & 497  & MM \\
      1124 & Kabre             & 27.63333 & 86.13333 & 1755 & MM \\
      1212 & Phatepur          & 26.73054 & 86.93481 & 101  & TAR \\
      1219 & Salleri           & 27.50512 & 86.58622 & 2383 & HM \\
      1222 & Diktel            & 27.21252 & 86.79189 & 1612 & MM \\
      1304 & Pakhribas         & 27.04632 & 87.29247 & 1720 & MM \\
      1327 & Khadbari          & 27.39106 & 87.20438 & 1064 & MM \\
      1303 & Chainpur (East)   & 27.2921  & 87.31697 & 1277 & MM \\
      1307 & Dhankuta          & 26.98322 & 87.34596 & 1192 & MM \\
      1024 & Dhulikhel         & 27.61612 & 85.5655  & 1543 & MM \\
      1419 & Phidim (Panchther) & 27.14367 & 87.7656  & 1157 & MM \\
      1314 & Terhathum         & 27.12304 & 87.53619 & 1525 & MM \\
      XXXX & Lubuche           & 27.96111 & 86.80889 & 5200 & H \\
      \bottomrule % Replaces the final \hline
  \end{tabularx}
  \begin{flushleft}
      \footnotesize{a Station numbers correspond to Department of Hydrology and Meteorology station index numbers. Numbers increase from west to east, and north to south.} \\
      \footnotesize{b TAR stands for Terai, SW stands for Siwalik, MM represents Middle Mountain, HM stands for High Mountains, and H stands for Himalaya regions}
  \end{flushleft}
\end{table}

\subsection{Data and Methods}


Daily maximum and minimum temperature measurements from synoptic, aero-synoptic, and agrometeorological stations throughout the Koshi basin for a 30-year period (1962–2022) were provided for this research by the Department of Hydrology and Meteorology (DHM) in Nepal. Twenty-three temperature stations from 101 m asl to 5200 m asl (Table \ref{tab:station_descriptions}) were selected for this study. Python version 3.12 is used for the data analysis. Pandas, Numpy, sklearn, pymannkendall, scipy, pykrig are used for data analysis; geopandas, matplotlib, shpely are used for graphical representation. The study data exhibited no significant inhomogeneity, although some data points were missing. Missing data were calculated and filled by lapse rate formula:

\begin{equation}
  T_{\text{cal}} = T_{\text{obs}} + (H_{\text{Elevation}} - L_{\text{Elevation}}) \cdot (-0.0065)
  \label{eq:lapse_calculation}
  \end{equation}
  
  Where, 
  \begin{align*}
  T_{\text{cal}} & = \text{High elevation calculating temperature} \\
  T_{\text{obs}} & = \text{Low elevation observed temperature} \\
  H_{\text{Elevation}} & = \text{High elevation (calculating temperature station's elevation)}  \\
  L_{\text{Elevation}} & = \text{Low elevation (Observed temperature station's elevation)}   
  \end{align*}

Annual and seasonal averages were computed for each year across all stations. The seasonal categorization was as follows: winter (including December of the previous year, January, and February), pre-monsoon (March to May), monsoon (June to September), and post-monsoon (October and November). Physiographically, Nepal is divided into five regions: Terai, Siwalik, Middle Mountains, High Mountains, and Himalayas \parencite{nayava_spatial_2017}, In line with this classification, this study also considers five physiographic regions to analyze the spatial variations of temperatures in the Koshi Basin. Seasonal and annual temperature trends for all stations were analyzed using linear regression. The spatial distribution of these temperature trends was mapped through interpolation, utilizing Kriging based on the station trends.

\textcite{chand_trend_2019} calculated a lapse rate of \(0.006 \ ^\circ \mathrm{C} \, \mathrm{m}^{-1}\) in the Narayani River Basin, while \textcite{nayava_spatial_2017} found a lapse rate of \(0.0058 \ ^\circ \mathrm{C} \, \mathrm{m}^{-1}\) for eastern Nepal and \(0.0057 \ ^\circ \mathrm{C} \, \mathrm{m}^{-1}\) for the entire country. This research utilized a theoretical lapse rate value of \(0.0065 \ ^\circ \mathrm{C} \, \mathrm{m}^{-1}\).

Spatial and temporal variations in air temperature are influenced by factors such as physiography (e.g., slope, aspect, hilltops, and valleys), land cover characteristics, and incoming solar radiation therefore Quantification of the contribution of each factor is complicated. This study has focused solely on temperature variations based on altitude, specifically within physiographic regions. Three types of temperature data were analyzed: Tmax, the mean of daily maximum temperature; Tmin, the mean of daily minimum temperature; and Tavg, the daily average temperature. To compare the relative magnitudes of the temperature data, the \textcite{mann_nonparametric_1945,kendall_rank_1949} test was employed to estimate monotonic trends—whether positive or negative—and their statistical significance. This analysis was conducted using (equation \ref{eq:sum})  and (equation \ref{eq:sgn}), with the variance determined through (equation \ref{eq:variance}):

\begin{equation} 
  S = \sum_{i=1}^{n-1} \sum_{j=i+1}^{n} \text{sgn}(x_j - x_i) \tag{1} 
  \label{eq:sum}
  \end{equation}
  
\begin{equation}
  \text{sgn}(x_j - x_i) = 
  \begin{cases} 
  1 & \text{if } (x_j - x_i) > 0 \\ 
  0 & \text{if } (x_j - x_i) = 0 \\ 
  -1 & \text{if } (x_j - x_i) < 0 
  \end{cases} \tag{2} 
  \label{eq:sgn}
  \end{equation}
  
\begin{equation}
  \text{Var}(S) = \frac{n(n-1)(2n+5) - \sum_{i=1}^{m} t_i(t_i-1)(2t_i+5)}{18} \tag{3}
  \label{eq:variance}
  \end{equation}

Where \( n \) is the number of data points, \( m \) is the number of tied groups, and \( t_i \) is the number of observations in the \( i^{th} \) tied group.

The collected data were thoroughly checked and screened for quality and consistency. To analyze the temporal and spatial variations of temperature, linear regression was employed. A time series analysis was conducted. This technique allowed the fitting of a linear trend between the time series data (y) and time (t), as described by the equation \ref{eq:linear_regression}:

\begin{equation}
  y = a + bt \tag{4}
\label{eq:linear_regression}
\end{equation}

Where:
\begin{itemize}
  \item \( y \): represents the temperature or rainfall,
  \item \( t \): represents time (in years),
  \item \( a \) and \( b \) are constants estimated using the least squares method, which minimizes the sum of the squared differences between the observed and predicted values.
\end{itemize}

This method provides an accurate and reliable estimation of the overall trend of temperature data across the study period.


