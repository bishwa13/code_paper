Nepal, a landlocked mountainous country in South Asia, is situated between 26$^\circ$22$'$–30$^\circ$27$'$ N latitude and 80$^\circ$04$'$–88$^\circ$12$'$ E longitude. Nestled in the southern foothills of the Himalayas, Nepal shares its northern border with China, while India surrounds it on the other three sides \parencite{ningLandUseCover}. Out of the total 147,181 square kilometers land area of Nepal \parencite{timilsinaAGRICULTURELANDUSE2019} and has a population of approximately 28.9 million \parencite{ningLandUseCover}. Nepal's land cover was primarily dominated by forests, croplands, and grasslands. Forests accounted for 49.9\% of the total land area, followed by croplands at 29.2\% and grasslands at 15.6\%. Over time, land-use patterns have undergone significant transformations, with notable shifts between different land types \parencite{ningLandUseCover}. Agriculture is the backbone of Nepal's economy, providing livelihoods for the majority of its population and playing a critical role in ensuring food security. However, the sector faces numerous challenges, including climate change, land degradation, and low productivity.Agriculture is the backbone of Nepal's economy, providing livelihoods for the majority of its population and playing a critical role in ensuring food security. However, the sector faces numerous challenges, including climate change, land degradation, and low productivity. and top of that only the 75\% of cropland is cultivated \parencite{timilsinaAGRICULTURELANDUSE2019}. In Nepal, the availability of arable land is limited, and the proportion of cultivated land per capita remains low. With a growing population, there is immense pressure to enhance agricultural productivity to meet rising food demands. This situation necessitates the adoption of effective land management strategies and sustainable agricultural practices to maximize crop yields while preserving soil health \parencite{rawalSoilNutrientBalance2022}.

It is increasingly evident that climate change is a tangible phenomenon, with fluctuations in temperature and precipitation affecting nutrient levels, soil moisture conditions, and, consequently, crop production. These impacts can be particularly severe if they coincide with critical growth stages of crops, potentially leading to devastating yield losses. To mitigate such adverse effects, the adoption of adaptive management practices is essential. Effective implementation of these practices requires a comprehensive understanding of appropriate fertilizer dosages, irrigation regimes, and soil nutrient dynamics \parencite{shresthaAssessmentClimateChange2017}. The lack of precise information on soil nutrient levels often results in the overuse or underuse of fertilizers, causing imbalanced nutrient supply, reduced productivity, and progressive soil degradation.

Soils in both natural ecosystems and agricultural landscapes exhibit inherent heterogeneity, resulting from a complex interplay of geochemical processes and anthropogenic influences. Agricultural activities, such as crop cultivation and soil management practices, further contribute to this variability by altering soil properties across spatial scales. Consequently, farm soils display inconsistent fertility and productivity levels, driven by differences in soil parameters and the availability of natural resources. However, the lack of detailed information regarding the levels and spatial distribution of soil nutrients at specific locations and fields often leads to the application of uniform fertilizer recommendations. This blanket approach frequently causes imbalanced nutrient supply, diminished crop productivity, and progressive soil degradation. To mitigate these issues, assessing the spatial distribution of soil nutrients is imperative for enabling precise nutrient management, optimizing crop yields, and fostering sustainable agricultural practices \parencite{dahalNewApproachMeasure2024a}.
A large proportion of the population engaged in agriculture follows a subsistence farming system. Only 25.1\% of farmers are engaged in commercial farming, while the remaining 75.9\% practice subsistence farming \parencite{simkhadaReviewNepalsIncreasing2019}. Agriculture in Nepal is based on subsistence farming for the majority of the population and this can never be underestimated. The growth of Nepalese agriculture was low and very vulnerable in recent decades  \parencite{gyawaliOverviewAgricultureNepal2021}. Despite the excellent production potential, farmers continue to face enormous challenges. Poverty, land degradation, low agricultural productivity, wrong use of budget and subsidies, shortage. The development of agriculture was hindered by the number of agricultural inputs, poor government support, etc. The challenge of maintaining food security was the simultaneous growth of the population on the one hand and the reduction of cultivated land on the other \parencite{gyawaliOverviewAgricultureNepal2021}.

This study aims to address this gap by assessing the spatial distribution of soil nutrients in Janaki Rural Municipality, Banke, Nepal. By analyzing soil nutrient dynamics and their impact on crop yields, this research provides actionable insights for optimizing fertilizer use, enhancing crop productivity, and promoting sustainable agricultural practices.


