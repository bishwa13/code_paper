Agriculture in Nepal is based on subsistence farming for the majority of the population and this can never be underestimated. The growth of Nepalese agriculture was low and very vulnerable in recent decades  \parencite{gyawaliOverviewAgricultureNepal2021}. Despite the excellent production potential, farmers continue to face enormous challenges. Poverty, land degradation, low agricultural productivity, wrong use of budget and subsidies, shortage. The development of agriculture was hindered by the number of agricultural inputs, poor government support, etc. The challenge of maintaining food security was the simultaneous growth of the population on the one hand and the reduction of cultivated land on the other \parencite{gyawaliOverviewAgricultureNepal2021}.
In Nepal's agriculture, coarse grains and pulses are crucial for food and nutritional security, the nitrogen economy, and job possibilities  They serve a dual purpose in Nepal, playing a significant role in both trade and survival. Coarse grains and pulses are important components of the Nepalese diet, providing much-needed nutrients to the common people(Db et al., 2014). 
1.1.2.2 Current Scenario:
Out of Nepal's total land area of 147,181 square kilometers, cultivated agricultural land makes up 21\% and uncultivated agriculture land makes 7\% of total land use (MOALD, 2020). In the last few years, Nepal's agricultural growth has accelerated significantly. The remaining 75.9\% of farmers do subsistence farming, while just 25.1\% engage in commercial farming. Although grain, cash crop, and pulse output has greatly improved over the past few years, it still isn't enough to feed the growing population (Simkhada 2019). The majority of farms in Nepal (53\%) are small-scale, with landholdings of no more than 0.5 hectares, while the remaining 20\% are large-scale, with landholdings of more than 1 hectares. More than one-fourth (26\%) of agricultural landowners in Nepal are women (FAQ 2020). 

\subsubsection{Agricultural Land Use}
The land problem is one of the most common concerns of all nations. In Nepal, cultivation is still unplanned and uncontrolled and there has been a tremendous shift in land use from agriculture to other purposes where cultivation is more flourishing in the Terai region of Nepal. Demand for land-intensive crops is increasing in the rapidly developing South Asian region, including Nepal (Timilsina et al., 2019). A recent FRTC study (2019) showed that the cultivated land is 21.88 percent of the entire country. This shows the recent growth of cultivated area in Nepal. The average land holding per family in Nepal is less than 0.68 ha (CBS, 2013), which has gradually decreased over the last three decades. In addition, agricultural land is regularly fragmented for several reasons, especially in the Terai zone of Nepal (Shrestha, 2011). Land use for temporary and permanent crops grown show an opposite trend, with permanent crops increasing and temporary crops decreasing. The number of land parcels has significantly increased, supporting the fact about increased land fragmentation trend in Nepal.
Today's growth in crop production in Nepal is due only to increased production area, however, the trend is to reduce agricultural land especially in the Terai due to urbanization and land fragmentation between 1989 and 2016 is a major obstacle to achieving food security in the country. So, the need is great increase the efficiency of land use to solve the problem of increasing food demand (Timilsina et al., 2019). 
2.5 Fertilizers in Agriculture:
More nutrients in the form of chemical fertilizers are needed to maximize high grain yields and meet the growing food demand of the majority of Nepal's population (Lamichhane et al., 2022). Chemical fertilizers are rich in nutrients; therefore, only a small amount is needed for productivity (Han et al 2016). A study carried by (Baral et al., 2020) in Nepalgunj, Banke found that different farm types, varietals, and irrigation systems have quite different fertilizer application procedures. Farmers utilized 55:39:15 kg N: P2O5:K2O ha-1 on average. The sources of these nutrients were urea, DAP, and MOP. Farmers' fertilizer usage was shown to be unbalanced overall. Amounts of N fertilizer used per hectare ranged from 0 kg to 138 kg. Compared to medium (57 kg ha-1) and small (41 kg ha-1) farmers, large farmers submitted applications at a greater rate (73 kg ha-1) (Baral et al., 2020). Although farmers have reported using more inorganic fertilizer over the previous five years, actual usage is still less than what the government advises. According to Takeshima et al. (2016), large farmers often use more nitrogen fertilizer than medium-sized and small-sized farms, perhaps because they have more financial resources and fewer organic inputs available.
2.8 Agricultural Land Use:
Demand for land-intensive crops is increasing in the rapidly developing South Asian region, including Nepal (Timilsina et al., 2019). A recent FRTC study (2019) showed that the cultivated land is 21.88 percent of the entire country. This shows the recent growth of cultivated area in Nepal. The average land holding per family in Nepal is less than 0.68 ha (CBS, 2013), which has gradually decreased over the last three decades. In addition, agricultural land is regularly fragmented for several reasons, especially in the Terai zone of Nepal (Shrestha, 2011). 
Today's growth in crop production in Nepal is due only to increased production area, however, the trend is to reduce agricultural land especially in the Terai due to urbanization and land fragmentation between 1989 and 2016 is a major obstacle to achieving food security in the country. So, the need is great increase the efficiency of land use to solve the problem of increasing food demand (Timilsina et al., 2019). According to (Devkota et al., 2023) Nepalgunj had the highest coverage of agricultural areas; the proportion of agricultural areas in Nepalgunj was 95.38\% (83.51 km2), 90.03\% (78.82 km2) and 89.77\% (78.6 km2) in the year 1990, 2000 and 2010 respectively.

This study focuses on understanding the impact of climate change on agriculture and food security in Janaki Rural Municipality, Banke, with a particular emphasis on nutrient dynamics, fertilizer use, and land-use changes. Climate change has disrupted soil nutrient availability and fertilizer efficiency, thereby affecting crop yields. Additionally, shifting land-use patterns, such as the conversion of agricultural lands for non-agricultural purposes, have further diminished the area available for cultivation. Analyzing the status of soil fertility and irrigation, along with changes in cultivated land over the past 20 years, is critical to addressing the challenges posed by climate change. By evaluating these factors and their relationship to food security, the study aims to provide insights that will guide sustainable agricultural practices and strategic planning to mitigate climate impacts on food production in the region.

1.1.5.2 Fertilizers/ Nutrient:
More nutrients in the form of chemical fertilizers are needed to maximize high grain yields and meet the growing food demand of the majority of Nepal's population (Lamichhane et al., 2022).One of the most important inputs for agricultural output is fertilizer. According to APP, higher fertilizer usage accounts for nearly half of the incremental production and it has been determined that a key contributing factor in the poor production and productivity of agricultural commodities is farmers' inadequate access to seed and fertilizer (Pandey et al., 2017). Chemical fertilizers are used to correct plant nutrient deficiencies, providing large amounts of nutrients to help plants withstand stress, maintain optimal soil fertility, and improve crop quality. Because the nutrients in chemical fertilizers are already soluble in water, they promote the rapid development of plants and are effective quickly and effectively. Chemical fertilizers are rich in nutrients; therefore, only a small amount is needed for productivity (Han et al 2016). Chemical fertilizers have been found to increase yields for a few years, but are ineffective in the long term and contribute to soil degradation. Chemical fertilizers accelerate the decomposition of organic matter in the soil, which weakens the soil structure and reduces soil aggregation. As a result, nutrients are easily lost from the soil through fixation, leaching and gassing, reducing the effectiveness of fertilizers (Alimi et al 2007). Higher doses of chemical fertilizers not only reduce the microbial population (Gruhn et al 2000) but also causes an imbalance of soil nutrients, which can lead to soil acidity and reduced yields (Ojeniyi 2002).

