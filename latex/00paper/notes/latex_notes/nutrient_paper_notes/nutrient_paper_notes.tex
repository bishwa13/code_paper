\documentclass[a4paper,12pt]{article}
\usepackage[utf8]{inputenc}
\usepackage{geometry}
\geometry{margin=1in}
\usepackage{amsmath, amssymb} 
\usepackage[backend=biber, style=apa]{biblatex}
\usepackage{adjustbox}
\usepackage{authblk}
\usepackage{array}
\usepackage{amsmath}
\usepackage{amssymb}

\usepackage{booktabs}

\usepackage{caption}

\usepackage{hyperref} 
\usepackage{setspace}
\usepackage{geometry}
\usepackage{graphicx}
\usepackage{tabularx}



\usepackage{float}
\usepackage{subcaption}
\usepackage{multirow}
\usepackage{hyperref}
\usepackage{xcolor}
\usepackage{titlesec}
\titleformat{\section}{\normalfont\Large\bfseries}{\thesection}{1em}{\MakeUppercase}

\addbibresource{nutrient_paper_notes.bib}
\title{Journal Paper Reading Notes on Nutrient Paper}
\author{Bishwa}
\date{\today}
\begin{document}

\maketitle



\section{AGRICULTURE LAND USE IN NEPAL: PROSPECTS AND IMPACTS ON FOOD SECURITY}
\parencite{timilsinaAGRICULTURELANDUSE2019}
Out of the total 147,181 square kilometers land area of Nepal, agricultural land is 28 percent (of which 21 percent is cultivated and 7 percent uncultivated); forest area is about 40 percent and pasture covers 12 percent.

AI: The remaining 20 percent is covered by snow, desert, and urban areas. The total population of Nepal is 26.5 million, and the population density is 180 per square kilometer. The average farm size is 0.6 hectares. The agriculture sector contributes 33 percent to the GDP and employs 66 percent of the total labor force. The agriculture sector is the main source of food, income, and employment for the majority of the population. The agriculture sector is characterized by low productivity, subsistence farming, and traditional farming practices. The agriculture sector is facing several challenges, such as low productivity, land degradation, water scarcity, climate change, and food insecurity. The agriculture sector is also facing several opportunities, such as the availability of arable land, water resources, and favorable agro-climatic conditions. The agriculture sector has the potential to increase productivity, reduce poverty, and ensure food security. The agriculture sector can play a crucial role in achieving the Sustainable Development Goals (SDGs) and the national development goals of Nepal. The agriculture sector can contribute to economic growth, poverty reduction, and food security. The agriculture sector can also contribute to environmental sustainability, climate change adaptation, and disaster risk reduction. The agriculture sector can contribute to the conservation of biodiversity, natural resources, and ecosystem services. The agriculture sector can contribute to the


\begin{table}[htbp]
    \centering
    \caption{Percentage of Farmer Households Affected by Climate Change in Banke District and Janaki Rural Municipality}
    \resizebox{\textwidth}{!}{
    \begin{tabular}{l c c c c c c c c c c c c}
    \toprule
    \multirow{2}{*}{Local Level} & \multirow{2}{*}{Total Households} & Aware of Climate Change & Reporting Impact & \multicolumn{9}{c}{Specific Impacts Reported (\% of those reporting impact)} \\
    \cmidrule(lr){5-13}
    & & (\% of total) & (\% of total) & Reduction in Prod. & Increase in Prod. & Change in Size & Change in Taste & Change in Planting & Rainfall Variation & Pest Outbreak & Breeding Change & Species Change \\
    \midrule
    Banke District & 67,885 & 43.0\% & 38.2\% & 81.6\% & 18.5\% & 28.5\% & 32.0\% & 36.6\% & 52.2\% & 55.7\% & 5.8\% & 8.5\% \\
    Janaki Rural Municipality & 6,274 & 30.7\% & 22.3\% & 54.5\% & 3.2\% & 26.6\% & 14.2\% & 29.0\% & 43.2\% & 45.0\% & 0\% & 0\% \\
    \bottomrule
    \end{tabular}
    }
    \label{tab:climate_change_CBS_results}
    \end{table}
\parencite{cbsNationalAgriculturalSurvey20212023}
CBS. (2023). National Agricultural Survey of Nepal-2021. Government of Nepal, National Planning Commission Secretariat. Central Bureau of Statistics, Kathmandu.






























\printbibliography
\end{document}