\documentclass[a4paper,12pt]{article}
\usepackage[utf8]{inputenc}
\usepackage{geometry}
\geometry{margin=1in}
\usepackage{amsmath, amssymb} 
\usepackage[backend=biber, style=apa]{biblatex}
\usepackage{adjustbox}
\usepackage{authblk}
\usepackage{array}
\usepackage{amsmath}
\usepackage{amssymb}

\usepackage{booktabs}

\usepackage{caption}

\usepackage{hyperref} 
\usepackage{setspace}
\usepackage{geometry}
\usepackage{graphicx}
\usepackage{tabularx}



\usepackage{float}
\usepackage{subcaption}
\usepackage{multirow}
\usepackage{hyperref}
\usepackage{xcolor}
\usepackage{titlesec}
\titleformat{\section}{\normalfont\Large\bfseries}{\thesection}{1em}{\MakeUppercase}

\addbibresource{nutrient_paper_notes.bib}
\title{Journal Paper Reading Notes on Nutrient Paper}
\author{Bishwa}
\date{\today}
\begin{document}

\maketitle



\section{AGRICULTURE LAND USE IN NEPAL: PROSPECTS AND IMPACTS ON FOOD SECURITY}
\parencite{timilsinaAGRICULTURELANDUSE2019}
Out of the total 147,181 square kilometers land area of Nepal, agricultural land is 28 percent (of which 21 percent is cultivated and 7 percent uncultivated); forest area is about 40 percent and pasture covers 12 percent.

AI: The remaining 20 percent is covered by snow, desert, and urban areas. The total population of Nepal is 26.5 million, and the population density is 180 per square kilometer. The average farm size is 0.6 hectares. The agriculture sector contributes 33 percent to the GDP and employs 66 percent of the total labor force. The agriculture sector is the main source of food, income, and employment for the majority of the population. The agriculture sector is characterized by low productivity, subsistence farming, and traditional farming practices. The agriculture sector is facing several challenges, such as low productivity, land degradation, water scarcity, climate change, and food insecurity. The agriculture sector is also facing several opportunities, such as the availability of arable land, water resources, and favorable agro-climatic conditions. The agriculture sector has the potential to increase productivity, reduce poverty, and ensure food security. The agriculture sector can play a crucial role in achieving the Sustainable Development Goals (SDGs) and the national development goals of Nepal. The agriculture sector can contribute to economic growth, poverty reduction, and food security. The agriculture sector can also contribute to environmental sustainability, climate change adaptation, and disaster risk reduction. The agriculture sector can contribute to the conservation of biodiversity, natural resources, and ecosystem services. The agriculture sector can contribute to the


Soil Nutrient Dynamics in Nepal

Soil nutrient dynamics play a crucial role in determining crop yields in Nepal, particularly in the context of the rice-wheat cropping system. The inherent potassium (K) levels in Nepalese soils have historically been high, but recent studies indicate a depletion of K levels, necessitating increased K fertilizer application to sustain crop yields. Field trials conducted across various agro-ecozones in Nepal suggest that increasing K application rates significantly enhances grain yields, with optimal rates varying by region (Ojha et al., 2021). Similarly, nutrient use efficiency (NUE) studies in wheat have shown that balanced NPK fertilization is essential for maximizing yields and maintaining soil fertility (Rawal et al., 2022).

Impacts on Crop Yields

The impact of soil nutrient management on crop yields is evident in several studies. For instance, long-term experiments have demonstrated that while nitrogen (N) application increases yields, the addition of phosphorus (P) and K does not always result in yield improvements, indicating that these nutrients are not always limiting factors (Gami et al., 2001). However, the depletion of soil K over time has been identified as a primary reason for declining yields in rice and wheat, highlighting the need for adequate K fertilization (Regmi et al., 2002). Additionally, the incorporation of organic matter, such as farmyard manure, has been shown to improve soil fertility and maintain crop yields over extended periods (Gami et al., 2001).

Nutrient Management Strategies

Effective nutrient management strategies are crucial for sustaining soil fertility and crop productivity. Studies emphasize the importance of balanced fertilization, incorporating both organic and inorganic nutrient sources, to maintain soil nutrient balance and prevent long-term depletion (Rawal et al., 2022). The use of integrated nutrient management techniques, such as residue incorporation and the application of farmyard manure, has been recommended to enhance soil fertility and increase farm income, particularly in the mid-hills of Nepal (Tiwari et al., 2010). Moreover, the adoption of transition season management practices, such as the incorporation of wheat straw and cultivation of short-cycled crops, can reduce nitrogen losses and improve grain yields (Becker et al., 2007).

Conclusion

In summary, the dynamics of soil nutrients, particularly potassium, play a significant role in influencing crop yields in Nepal. While nitrogen application is crucial for yield improvement, the depletion of potassium and phosphorus over time necessitates careful management to sustain productivity. Integrated nutrient management strategies, including the use of organic amendments and balanced fertilization, are essential for maintaining soil fertility and enhancing crop yields in Nepal's diverse agro-ecozones.

These papers were sourced and synthesized using Consensus, an AI-powered search engine for research. Try it at https://consensus.app

References
$
Ojha, R., Shrestha, S., Khadka, Y., & Panday, D. (2021). Potassium nutrient response in the rice-wheat cropping system in different agro-ecozones of Nepal. PLoS ONE, 16. https://doi.org/10.1371/journal.pone.0248837

Rawal, N., Pande, K., Shrestha, R., & Vista, S. (2022). Nutrient use efficiency (NUE) of wheat (Triticum aestivum L.) as affected by NPK fertilization. PLoS ONE, 17. https://doi.org/10.1371/journal.pone.0262771

Rawal, N., Pande, K., Shrestha, R., & Vista, S. (2022). Soil Nutrient Balance and Soil Fertility Status under the Influence of Fertilization in Maize-Wheat Cropping System in Nepal. Applied and Environmental Soil Science. https://doi.org/10.1155/2022/2607468

Gami, S., Ladha, J., Pathak, H., Shah, M., Pasuquin, E., Pandey, S., Hobbs, P., Joshy, D., & Mishra, R. (2001). Long-term changes in yield and soil fertility in a twenty-year rice-wheat experiment in Nepal. Biology and Fertility of Soils, 34, 73-78. https://doi.org/10.1007/s003740100377

Regmi, A., Ladha, J., Pathak, H., Pasuquin, E., Bueno, C., Dawe, D., Hobbs, P., Joshy, D., Maskey, S., & Pandey, S. (2002). Yield and Soil Fertility Trends in a 20‐Year Rice–Rice–Wheat Experiment in Nepal. Soil Science Society of America Journal, 66, 857-867. https://doi.org/10.2136/SSSAJ2002.8570

Becker, M., Asch, F., Maskey, S., Pande, K., Shah, S., & Shrestha, S. (2007). Effects of transition season management on soil N dynamics and system N balances in rice-wheat rotations of Nepal. Field Crops Research, 103, 98-108. https://doi.org/10.1016/J.FCR.2007.05.002

Tiwari, K., Sitaula, B., Bajracharya, R., & Børresen, T. (2010). Effects of soil and crop management practices on yields, income and nutrients losses from upland farming systems in the Middle Mountains region of Nepal. Nutrient Cycling in Agroecosystems, 86, 241-253. https://doi.org/10.1007/s10705-009-9289-0



$


























\printbibliography
\end{document}