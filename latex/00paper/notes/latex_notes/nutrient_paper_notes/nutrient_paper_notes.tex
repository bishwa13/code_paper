\documentclass[a4paper,12pt]{article}
\usepackage[utf8]{inputenc}
\usepackage{geometry}
\geometry{margin=1in}
\usepackage{amsmath, amssymb} 
\usepackage[backend=biber, style=apa]{biblatex}
\usepackage{adjustbox}
\usepackage{authblk}
\usepackage{array}
\usepackage{amsmath}
\usepackage{amssymb}

\usepackage{booktabs}

\usepackage{caption}

\usepackage{hyperref} 
\usepackage{setspace}
\usepackage{geometry}
\usepackage{graphicx}
\usepackage{tabularx}



\usepackage{float}
\usepackage{subcaption}
\usepackage{multirow}
\usepackage{hyperref}
\usepackage{xcolor}
\usepackage{titlesec}
\titleformat{\section}{\normalfont\Large\bfseries}{\thesection}{1em}{\MakeUppercase}

\addbibresource{nutrient_paper_notes.bib}
\title{Journal Paper Reading Notes on Nutrient Paper}
\author{Bishwa}
\date{\today}
\begin{document}

\maketitle

\section{Journal Idea}
Journal of Agriculture and Food Research  journal homepage: www.sciencedirect.com/journal/journal-of-agriculture-and-food-research

\section{Farmers’ Fertilizer Application Gap In Ricebased Cropping System: A Case Studyof Nepal}
\parencite{baralFarmersFertilizerApplication2020}
Although farmers have reported a rising trend in the use of inorganic fertilizers over recent years, the overall application rates remain below the levels recommended by government guidelines. Notably, large-scale farmers tend to apply higher quantities of nitrogen (N) fertilizer compared to their medium- and small-scale counterparts, reflecting disparities in resource access or farming practices across farm sizes.


\section{Assessment of climate change impact on crop yield and irrigation water requirement of two major cereal crops (rice and wheat) in Bhaktapur district, Nepal}

\parencite{shresthaAssessmentClimateChange2017}
%It is increasingly evident that climate change is a tangible phenomenon, with fluctuations in temperature and precipitation affecting nutrient levels, soil moisture conditions, and, consequently, crop production. These impacts can be particularly severe if they coincide with critical growth stages of crops, potentially leading to devastating yield losses. To mitigate such adverse effects, the adoption of adaptive management practices is essential. Effective implementation of these practices requires a comprehensive understanding of appropriate fertilizer dosages, irrigation regimes, and soil nutrient dynamics.

A study conducted in Bhaktapur district by \parencite{shresthaAssessmentClimateChange2017} revealed that the organic matter content in soil samples was consistently below 3\%. The research highlighted that increasing organic matter enhances the soil’s water-holding capacity and hydraulic conductivity, thereby improving its resilience to environmental stress.
Based on these findings, the study recommends several strategies to stabilize crop yields under changing climatic conditions. These include adjusting crop planting dates, adopting temperature-resilient crop genotypes to mitigate heat stress, and optimizing water and fertilizer management practices to ensure sustainable agricultural productivity.




\section{Soil organic matter content and crop yield}
\parencite{lalSoilOrganicMatter2020}
An increase in soil organic matter (SOM) content can significantly influence crop yield by serving as a source of plant nutrients, particularly in nutrient-limited environments where chemical fertilizer inputs are minimal. This contribution is especially valuable in sustaining productivity under conditions of low external nutrient supplementation, as SOM enhances soil fertility through the gradual release of essential elements.
Beyond its role in nutrient supply, SOM contributes to crop production through additional mechanisms. These include improvements in soil structure, water retention, and microbial activity, which collectively enhance the growing environment and support plant development, independent of its nutrient provisioning capacity.
A study by \parencite{lalSoilOrganicMatter2020} concluded that elevating SOM levels in depleted or degraded soils positively impacts crop yield. The research further demonstrated that increasing SOM content can reduce the dependency on nitrogenous fertilizers and irrigation water, offering a resource-efficient alternative for soil fertility management. However, the study noted that the application of fertilizers and irrigation can obscure the beneficial effects of SOM, masking its contributions to crop productivity under high-input conditions.




\section{A new approach to measure spatial variability of soil parameters and field  technique to test-value specific fertilizer recommendations}
\parencite{dahalNewApproachMeasure2024a}
Soils in both natural ecosystems and agricultural landscapes exhibit inherent heterogeneity, resulting from a complex interplay of geochemical processes and anthropogenic influences. Agricultural activities, such as crop cultivation and soil management practices, further contribute to this variability by altering soil properties across spatial scales. Consequently, farm soils display inconsistent fertility and productivity levels, driven by differences in soil parameters and the availability of natural resources. However, the lack of detailed information regarding the levels and spatial distribution of soil nutrients at specific locations and fields often leads to the application of uniform fertilizer recommendations. This blanket approach frequently causes imbalanced nutrient supply, diminished crop productivity, and progressive soil degradation. To mitigate these issues, assessing the spatial distribution of soil nutrients is imperative for enabling precise nutrient management, optimizing crop yields, and fostering sustainable agricultural practices.

Effective site-specific nutrient management (SSNM) hinges on a comprehensive understanding of soil nutrient distributions. Insight into the spatial variability of soil properties enhances the ability to identify production constraints, determine the precise nutrient requirements of crops, and implement tailored nutrient management strategies. By accounting for localized differences in soil fertility, SSNM offers a targeted approach that improves resource efficiency and supports long-term agricultural sustainability.
A study conducted by Hari Dahal in Rangelee revealed significant variability in soil nutrient profiles. The research identified consistently low levels of nitrogen and organic matter, contrasted by elevated phosphorus concentrations, while potassium levels ranged from low to medium and high across the study area. These findings highlight the non-uniform nature of soil fertility, even within a single region, underscoring the limitations of generalized fertilizer applications.
Further analysis from the Rangelee study established nitrogen as the primary determinant of rice yield, emphasizing its critical role in crop performance. In contrast, phosphorus was found to have an insignificant impact on yield, while both phosphorus and potassium exhibited negative correlations with productivity. These results suggest that nutrient management strategies must prioritize nitrogen replenishment and carefully consider the potential adverse effects of excessive phosphorus and potassium, reinforcing the need for spatially informed approaches to soil fertility management.




\section{NAARC Soil Properties Summary}
\parencite{narc_soil_data}
\begin{table}[h]
    \centering
    \begin{tabular}{|l|c|c|c|c|}
    \hline
    \textbf{Properties}                  & \textbf{Max Value} & \textbf{Min Value} & \textbf{Mean Value} & \textbf{Median Value} \\ \hline
    pH value                             & 7.946             & 7.250             & 7.463              & 7.457                \\ \hline
    Organic Matter                       & 2.240 \%          & 1.529 \%          & 1.792 \%           & 1.790 \%             \\ \hline
    Total Nitrogen                       & 0.114 \%          & 0.076 \%          & 0.089 \%           & 0.090 \%             \\ \hline
    Available Phosphorus (P$_2$O$_5$)    & 252.324 kg/ha     & 136.833 kg/ha     & 181.041 kg/ha      & 178.447 kg/ha        \\ \hline
    Available Potassium (K$_2$O)         & 466.048 kg/ha     & 271.182 kg/ha     & 356.231 kg/ha      & 353.577 kg/ha        \\ \hline
    \end{tabular}
    \caption{Soil Properties Summary}
    \label{tab:soil_properties}
    \end{table}


\section{Land Use/Cover Change, Fragmentation, and Driving Factors in Nepal in the Last 25 Years}
\parencite{ningLandUseCover}
%Nepal, a landlocked mountainous country in South Asia, is situated between 26$^\circ$22$'$–30$^\circ$27$'$ N latitude and 80$^\circ$04$'$–88$^\circ$12$'$ E longitude. Nestled in the southern foothills of the Himalayas, Nepal shares its northern border with China, while India surrounds it on the other three sides. The country spans a total land area of 147,181 km$^2$ and has a population of approximately 28.9 million.

%In 2020, Nepal's land cover was primarily dominated by forests, croplands, and grasslands. Forests accounted for 49.9\% of the total land area, followed by croplands at 29.2\% and grasslands at 15.6\%. Over time, land-use patterns have undergone significant transformations, with notable shifts between different land types.

The area covered by coniferous and broadleaf forests has increased, primarily due to the conversion of croplands and shrublands into forested areas. Specifically, 1.5\% of coniferous forests expanded into former croplands, while 3.7\% of broadleaf forests emerged from croplands. However, this transformation has not been unidirectional, as 2.0\% of croplands were converted from forested areas. Additionally, approximately 47.4\% of construction land was developed on former cropland, highlighting the impact of human activities on land-use dynamics in Nepal.

\section{Review on Nepal’s Increasing Agricultural Import}
\parencite{simkhadaReviewNepalsIncreasing2019}
The agricultural fields of Nepal have lost soil fertility. The balanced use of chemical fertilizers and organic fertilizers in vegetable production can help farmers revitalize soil fertility and increase organic matter in the soil.

%A large proportion of the population engaged in agriculture follows a subsistence farming system. Only 25.1\% of farmers are engaged in commercial farming, while the remaining 75.9\% practice subsistence farming.





\section{AGRICULTURE LAND USE IN NEPAL: PROSPECTS AND IMPACTS ON FOOD SECURITY}
\parencite{timilsinaAGRICULTURELANDUSE2019}
%Out of the total 147,181 square kilometers land area of Nepal, agricultural land is 28 percent (of which 21 percent is cultivated and 7 percent uncultivated); forest area is about 40 percent and pasture covers 12 percent.

AI: The remaining 20 percent is covered by snow, desert, and urban areas. The total population of Nepal is 26.5 million, and the population density is 180 per square kilometer. The average farm size is 0.6 hectares. The agriculture sector contributes 33 percent to the GDP and employs 66 percent of the total labor force. The agriculture sector is the main source of food, income, and employment for the majority of the population. The agriculture sector is characterized by low productivity, subsistence farming, and traditional farming practices. The agriculture sector is facing several challenges, such as low productivity, land degradation, water scarcity, climate change, and food insecurity. The agriculture sector is also facing several opportunities, such as the availability of arable land, water resources, and favorable agro-climatic conditions. The agriculture sector has the potential to increase productivity, reduce poverty, and ensure food security. The agriculture sector can play a crucial role in achieving the Sustainable Development Goals (SDGs) and the national development goals of Nepal. The agriculture sector can contribute to economic growth, poverty reduction, and food security. The agriculture sector can also contribute to environmental sustainability, climate change adaptation, and disaster risk reduction. The agriculture sector can contribute to the conservation of biodiversity, natural resources, and ecosystem services. The agriculture sector can contribute to the

\section{Soil Nutrient Balance and Soil Fertility Status under the Influence of Fertilization in Maize-Wheat Cropping System in Nepal}
\parencite{rawalSoilNutrientBalance2022}
In Nepal, the availability of arable land is limited, and the proportion of cultivated land per capita remains low. With a growing population, there is immense pressure to enhance agricultural productivity to meet rising food demands. This situation necessitates the adoption of effective land management strategies and sustainable agricultural practices to maximize crop yields while preserving soil health.

Soil nutrient balance plays a crucial role in maintaining agricultural productivity, and it is significantly influenced by nutrient management practices. Poor nutrient management can lead to an imbalance in soil nutrients, which may have long-term negative consequences on crop production. Inappropriate fertilization practices, excessive use of chemical inputs, or nutrient depletion due to continuous cropping without replenishment can degrade soil fertility and reduce crop yields over time.

Crop response to nutrient application varies considerably depending on soil fertility levels and environmental conditions. The effectiveness of fertilizers is influenced by location-specific factors such as soil type, climate, and cropping patterns. In intensive cropping systems like the maize-wheat rotation, inadequate fertilizer management can lead to severe depletion of soil nutrients. Chemical fertilizers are essential components of modern agricultural systems, as they help fulfill crop nutrient requirements and support high-yield farming practices.

The role of chemical fertilizers in boosting agricultural production is well-documented. Studies have shown that commercial fertilizers can enhance crop yields by 30–50 percent, making them an indispensable part of modern farming. Their contribution to improving productivity has been instrumental in addressing food security challenges in many regions.

However, the prolonged and excessive application of chemical fertilizers can alter soil properties over time. Research indicates that continuous fertilizer use can lead to a decline in soil pH while increasing soil organic carbon content. Despite fertilizer applications, a significant reduction in the total nitrogen (N), phosphorus (P), and potassium (K) content has been observed in soils over time. This highlights the need for balanced fertilization strategies, integrated nutrient management approaches, and sustainable soil conservation practices to maintain long-term soil fertility and agricultural productivity.

\section{Nutrient use efficiency (NUE) of wheat  (Triticum aestivum L.) as affected by NPK  fertilization}
\parencite{rawalNutrientUseEfficiency2022}
Fertilizer applications are essential to maintain a positive nutrient balance by replenishing nutrients that are removed and lost during cropping.

\section{Integrated assessment of irrigation and agriculture management challenges in Nepal: An interdisciplinary perspective}
\parencite{nepalIntegratedAssessmentIrrigation2024}
Agriculture serves as a cornerstone of Nepal’s economy, playing a pivotal role in ensuring food and nutrition security, sustaining livelihoods, and providing rural employment. With a significant portion of the population dependent on farming, the sector’s stability is critical for national development. However, achieving sustainable agricultural productivity remains a persistent challenge, largely due to limitations in irrigation infrastructure and management, compounded by environmental and socioeconomic factors.
According to Nepal’s Irrigation Master Plan 2019, irrigation infrastructure currently covers approximately 50\% of the nation’s 2.5 million hectares of potentially irrigable land. This represents a substantial achievement in expanding access to water resources for agriculture. Nevertheless, significant gaps persist. There is a pressing need to further extend irrigated areas, enhance the efficiency of existing systems, and improve water productivity to bolster food security. These improvements are essential not only to meet current demands but also to address the growing pressures of population growth and climate variability, which threaten agricultural resilience.
Despite decades of investment in irrigation projects and institutional reforms aimed at strengthening water management, the performance of Nepal’s irrigation systems remains suboptimal. Studies indicate that many schemes suffer from inefficiencies, including inconsistent water delivery, outdated infrastructure, and inadequate maintenance. These shortcomings limit the potential for increased agricultural output and undermine efforts to achieve sustainable food production. Addressing these challenges requires an integrated approach that combines technical enhancements with socioeconomic and governance considerations, ensuring that irrigation development aligns with the broader goals of agricultural sustainability and rural prosperity.














































































\printbibliography
\end{document}