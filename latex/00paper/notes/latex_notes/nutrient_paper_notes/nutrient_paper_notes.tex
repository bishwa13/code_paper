\documentclass[a4paper,12pt]{article}
\usepackage[utf8]{inputenc}
\usepackage{geometry}
\geometry{margin=1in}
\usepackage{amsmath, amssymb} 
\usepackage[backend=biber, style=apa]{biblatex}
\usepackage{adjustbox}
\usepackage{authblk}
\usepackage{array}
\usepackage{amsmath}
\usepackage{amssymb}

\usepackage{booktabs}

\usepackage{caption}

\usepackage{hyperref} 
\usepackage{setspace}
\usepackage{geometry}
\usepackage{graphicx}
\usepackage{tabularx}



\usepackage{float}
\usepackage{subcaption}
\usepackage{multirow}
\usepackage{hyperref}
\usepackage{xcolor}
\usepackage{titlesec}
\titleformat{\section}{\normalfont\Large\bfseries}{\thesection}{1em}{\MakeUppercase}

\addbibresource{nutrient_paper_notes.bib}
\title{Journal Paper Reading Notes on Nutrient Paper}
\author{Bishwa}
\date{\today}
\begin{document}

\maketitle

\section{Journal Idea}
Journal of Agriculture and Food Research  journal homepage: www.sciencedirect.com/journal/journal-of-agriculture-and-food-research


\section{Land Use/Cover Change, Fragmentation, and Driving Factors in Nepal in the Last 25 Years}
\parencite{@ningLandUseCover}
Nepal, a landlocked mountainous country in South Asia, is situated between 26$^\circ$22$'$–30$^\circ$27$'$ N latitude and 80$^\circ$04$'$–88$^\circ$12$'$ E longitude. Nestled in the southern foothills of the Himalayas, Nepal shares its northern border with China, while India surrounds it on the other three sides. The country spans a total land area of 147,181 km$^2$ and has a population of approximately 28.9 million.

In 2020, Nepal's land cover was primarily dominated by forests, croplands, and grasslands. Forests accounted for 49.9\% of the total land area, followed by croplands at 29.2\% and grasslands at 15.6\%. Over time, land-use patterns have undergone significant transformations, with notable shifts between different land types.

The area covered by coniferous and broadleaf forests has increased, primarily due to the conversion of croplands and shrublands into forested areas. Specifically, 1.5\% of coniferous forests expanded into former croplands, while 3.7\% of broadleaf forests emerged from croplands. However, this transformation has not been unidirectional, as 2.0\% of croplands were converted from forested areas. Additionally, approximately 47.4\% of construction land was developed on former cropland, highlighting the impact of human activities on land-use dynamics in Nepal.

\section{Review on Nepal’s Increasing Agricultural Import}
\parencite{@simkhadaReviewNepalsIncreasing2019}
The agricultural fields of Nepal have lost soil fertility. The balanced use of chemical fertilizers and organic fertilizers in vegetable production can help farmers revitalize soil fertility and increase organic matter in the soil.

A large proportion of the population engaged in agriculture follows a subsistence farming system. Only 25.1\% of farmers are engaged in commercial farming, while the remaining 75.9\% practice subsistence farming.





\section{AGRICULTURE LAND USE IN NEPAL: PROSPECTS AND IMPACTS ON FOOD SECURITY}
\parencite{timilsinaAGRICULTURELANDUSE2019}
Out of the total 147,181 square kilometers land area of Nepal, agricultural land is 28 percent (of which 21 percent is cultivated and 7 percent uncultivated); forest area is about 40 percent and pasture covers 12 percent.

AI: The remaining 20 percent is covered by snow, desert, and urban areas. The total population of Nepal is 26.5 million, and the population density is 180 per square kilometer. The average farm size is 0.6 hectares. The agriculture sector contributes 33 percent to the GDP and employs 66 percent of the total labor force. The agriculture sector is the main source of food, income, and employment for the majority of the population. The agriculture sector is characterized by low productivity, subsistence farming, and traditional farming practices. The agriculture sector is facing several challenges, such as low productivity, land degradation, water scarcity, climate change, and food insecurity. The agriculture sector is also facing several opportunities, such as the availability of arable land, water resources, and favorable agro-climatic conditions. The agriculture sector has the potential to increase productivity, reduce poverty, and ensure food security. The agriculture sector can play a crucial role in achieving the Sustainable Development Goals (SDGs) and the national development goals of Nepal. The agriculture sector can contribute to economic growth, poverty reduction, and food security. The agriculture sector can also contribute to environmental sustainability, climate change adaptation, and disaster risk reduction. The agriculture sector can contribute to the conservation of biodiversity, natural resources, and ecosystem services. The agriculture sector can contribute to the

\section{Soil Nutrient Balance and Soil Fertility Status under the Influence of Fertilization in Maize-Wheat Cropping System in Nepal}
\parencite{@rawalSoilNutrientBalance2022}
In Nepal, the availability of arable land is limited, and the proportion of cultivated land per capita remains low. With a growing population, there is immense pressure to enhance agricultural productivity to meet rising food demands. This situation necessitates the adoption of effective land management strategies and sustainable agricultural practices to maximize crop yields while preserving soil health.

Soil nutrient balance plays a crucial role in maintaining agricultural productivity, and it is significantly influenced by nutrient management practices. Poor nutrient management can lead to an imbalance in soil nutrients, which may have long-term negative consequences on crop production. Inappropriate fertilization practices, excessive use of chemical inputs, or nutrient depletion due to continuous cropping without replenishment can degrade soil fertility and reduce crop yields over time.

Crop response to nutrient application varies considerably depending on soil fertility levels and environmental conditions. The effectiveness of fertilizers is influenced by location-specific factors such as soil type, climate, and cropping patterns. In intensive cropping systems like the maize-wheat rotation, inadequate fertilizer management can lead to severe depletion of soil nutrients. Chemical fertilizers are essential components of modern agricultural systems, as they help fulfill crop nutrient requirements and support high-yield farming practices.

The role of chemical fertilizers in boosting agricultural production is well-documented. Studies have shown that commercial fertilizers can enhance crop yields by 30–50 percent, making them an indispensable part of modern farming. Their contribution to improving productivity has been instrumental in addressing food security challenges in many regions.

However, the prolonged and excessive application of chemical fertilizers can alter soil properties over time. Research indicates that continuous fertilizer use can lead to a decline in soil pH while increasing soil organic carbon content. Despite fertilizer applications, a significant reduction in the total nitrogen (N), phosphorus (P), and potassium (K) content has been observed in soils over time. This highlights the need for balanced fertilization strategies, integrated nutrient management approaches, and sustainable soil conservation practices to maintain long-term soil fertility and agricultural productivity.

\section{Nutrient use efficiency (NUE) of wheat  (Triticum aestivum L.) as affected by NPK  fertilization}
\parencite{@rawalNutrientUseEfficiency2022}
Fertilizer applications are essential to maintain a positive nutrient balance by replenishing nutrients that are removed and lost during cropping.


































section--break

Soil Nutrient Dynamics in Nepal

Soil nutrient dynamics play a crucial role in determining crop yields in Nepal, particularly in the context of the rice-wheat cropping system. The inherent potassium (K) levels in Nepalese soils have historically been high, but recent studies indicate a depletion of K levels, necessitating increased K fertilizer application to sustain crop yields. Field trials conducted across various agro-ecozones in Nepal suggest that increasing K application rates significantly enhances grain yields, with optimal rates varying by region (Ojha et al., 2021). Similarly, nutrient use efficiency (NUE) studies in wheat have shown that balanced NPK fertilization is essential for maximizing yields and maintaining soil fertility (Rawal et al., 2022).

Impacts on Crop Yields

The impact of soil nutrient management on crop yields is evident in several studies. For instance, long-term experiments have demonstrated that while nitrogen (N) application increases yields, the addition of phosphorus (P) and K does not always result in yield improvements, indicating that these nutrients are not always limiting factors (Gami et al., 2001). However, the depletion of soil K over time has been identified as a primary reason for declining yields in rice and wheat, highlighting the need for adequate K fertilization (Regmi et al., 2002). Additionally, the incorporation of organic matter, such as farmyard manure, has been shown to improve soil fertility and maintain crop yields over extended periods (Gami et al., 2001).

Nutrient Management Strategies

Effective nutrient management strategies are crucial for sustaining soil fertility and crop productivity. Studies emphasize the importance of balanced fertilization, incorporating both organic and inorganic nutrient sources, to maintain soil nutrient balance and prevent long-term depletion (Rawal et al., 2022). The use of integrated nutrient management techniques, such as residue incorporation and the application of farmyard manure, has been recommended to enhance soil fertility and increase farm income, particularly in the mid-hills of Nepal (Tiwari et al., 2010). Moreover, the adoption of transition season management practices, such as the incorporation of wheat straw and cultivation of short-cycled crops, can reduce nitrogen losses and improve grain yields (Becker et al., 2007).

Conclusion

In summary, the dynamics of soil nutrients, particularly potassium, play a significant role in influencing crop yields in Nepal. While nitrogen application is crucial for yield improvement, the depletion of potassium and phosphorus over time necessitates careful management to sustain productivity. Integrated nutrient management strategies, including the use of organic amendments and balanced fertilization, are essential for maintaining soil fertility and enhancing crop yields in Nepal's diverse agro-ecozones.

These papers were sourced and synthesized using Consensus, an AI-powered search engine for research. Try it at https://consensus.app



























\printbibliography
\end{document}