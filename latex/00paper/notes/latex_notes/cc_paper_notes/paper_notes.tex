\documentclass[a4paper,12pt]{article}
\usepackage[utf8]{inputenc}
\usepackage{geometry}
\geometry{margin=1in}
\usepackage{amsmath, amssymb} 
\usepackage[backend=biber, style=apa]{biblatex}
\usepackage{adjustbox}
\usepackage{authblk}
\usepackage{array}
\usepackage{amsmath}
\usepackage{amssymb}

\usepackage{booktabs}

\usepackage{caption}

\usepackage{hyperref} 
\usepackage{setspace}
\usepackage{geometry}
\usepackage{graphicx}
\usepackage{tabularx}



\usepackage{float}
\usepackage{subcaption}
\usepackage{multirow}
\usepackage{hyperref}
\usepackage{xcolor}
\usepackage{titlesec}
\titleformat{\section}{\normalfont\Large\bfseries}{\thesection}{1em}{\MakeUppercase}

\addbibresource{paper_notes.bib}
\title{Journal Paper Reading Notes}
\author{Bishwa}
\date{\today}
\begin{document}

\maketitle




\section{Spatial and Temporal Variations of Surface Air Temperature (1962–2022) across Physiographic Regions in the Koshi Basin, Nepal}
\parencite{puriSpatialTemporalVariations2024}
As reported by \parencite{puriSpatialTemporalVariations2024}, annual regional temperature trends in the Koshi Basin over a 60-year period showed seasonal variability, but an overall slightly insignificant cooling trend was detected.

\section{AGRICULTURE LAND USE IN NEPAL: PROSPECTS AND IMPACTS ON FOOD SECURITY}
Cite this \parencite{timilsinaAGRICULTURELANDUSE2019}

\section{CROP YIELD RESPONSE TO CLIMATE CHANGE IN DIFFERENT ECOLOZICAL ZONES OF NEPAL}
According to the \parencite{regmiCROPYIELDRESPONSE2019} The average annual precipitation in Banke district over last 37 years is about 1395 mm with a range from 868 mm to 2173 mm per annum and shows a high deviation (Standard Deviation of more than 338). 
The average maximum temperature ranges from 26 to 31 with standard deviation of less than one and average minimum temperature ranges from 15 to 20 degrees Celsius again with standard deviation of less than one showing almost no change on its trend value.” 

“\parencite{regmiCROPYIELDRESPONSE2019} The effect of climate change on agriculture can be various forms, through changes in average temperature, rainfall and climate extremes. This study examines the crop yield response to climate changes using”

“The coefficients of maximum temperature and minimum temperature are statistically significant with wheat crop yield, where crop yield will likely decrease with the rise of maximum temperature but increase with the decrease of minimum temperature in Banke district.” 

“It is noteworthy to highlight that yields of paddy and maize did not show any impact by climatic variations in Banke. Perhaps this may be due to good irrigation facilities in Banke compared to other hill and mountain districts, leading to relatively low impact of climate change on irrigated land as compared rainfed land.” 

“Climate change affects crop productivity with a number of ways such as changes in temperature and precipitation which is being serious threat to crop productivity, thereby threatening to food security. The findings from the study of the effect of climate change variability on crop production show that overall rise in temperature” 

\section{SOCIOECONOMIC FACTORS AFFECTING AWARENESS AND ADAPTION OF CLIMATE CHANGE: A CASE STUDY OF BANKE DISTRICT NEPA}
\parencite{shresthaSOCIOECONOMICFACTORSAFFECTING2018b}
Research suggests older age, male gender, more farming experience, and access to mass media positively affect household heads' knowledge of climate change.

A primary adaptation strategy observed was the cultivation of tolerant varieties. Additionally, farmers employed temporal adjustments, including delayed planting in response to delayed monsoons and early planting during early monsoons. To mitigate yield losses and diversify risk, mixed or intercropping systems were utilized. Furthermore, supplemental irrigation, through pumps and boreholes, was implemented to address water scarcity. Finally, early wheat planting was practiced to avoid thermal stress during critical reproductive stages.
investigations reveal a complex interplay of socio-demographic factors and adaptive agricultural practices in the face of climate change. Notably, older age, male gender, greater farming experience, and enhanced access to mass media significantly augment household heads' climate change knowledge, a crucial precursor to effective adaptation. Concurrently, male-headed households exhibit a demonstrably higher probability of adopting climate change adaptation strategies, specifically than female-headed households, and increased farm size also correlates positively with adoption rates. Farmers primarily respond to climate variability through the cultivation of tolerant varieties, temporal adjustments in planting schedules, and risk diversification via mixed and intercropping. Supplemental irrigation and strategic planting adjustments, such as early wheat planting to mitigate thermal stress, further exemplify their adaptive capacity. These findings underscore the necessity of targeted interventions that consider both the knowledge determinants and the practical adaptive strategies employed by farming communities, with particular attention to gender disparities and farm size influences.

\section{The Relationships between Climate Variability and Crop Yield in a Mountainous Environment: A Case Study in Lamjung District, Nepal}
\parencite{poudelRelationshipsClimateVariability2016}

Paragraph 1: Climate Trends:

"\parencite{poudelRelationshipsClimateVariability2016} Analysis of climatic data from Lamjung District, a mountainous region in Nepal, revealed heterogeneous precipitation trends. Two stations exhibited increasing precipitation, while one showed a decreasing trend, likely due to the complex interplay of monsoon and westerly wind systems across the varied topography. Temperature analysis indicated a consistent warming trend, with an average increase of 0.07 degrees Celsius."

Paragraph 2: Crop Yield Dynamics:

" \parencite{poudelRelationshipsClimateVariability2016}  Crop yield trends displayed significant temporal fluctuations, with an overall increasing pattern. This increase is likely attributable to a combination of factors, including climate change impacts and the adoption of improved agricultural practices. These practices encompass the introduction of new seed varieties, advanced agricultural technologies, enhanced irrigation systems, and refined crop management strategies."

Paragraph 3: Climate-Yield Relationship:

"A strong positive correlation was identified between climatic variability and the yields of millet and wheat. Conversely, rice, maize, and barley yields showed negligible or no relationship with climatic variations. This suggests differential crop sensitivity to climate fluctuations within the region."

Paragraph 4: Regression Analysis and Limitations:

"Multi-linear regression models explained varying degrees of yield variation, ranging from 0.372 for millet to 0.078 for rice. While the regression results demonstrated some significant relationships between yield and climate variables, the limited number of statistically significant associations highlights the complexity of yield determinants and suggests the need for further research to fully understand the climate-crop yield relationship in this region."


\section{Adapting cropping systems to climate change in Nepal: a cross-regional study of farmers’ perception and practices}

\parencite{manandharAdaptingCroppingSystems2011}
The meteorological station at Bhairahawa Airport indicates a diminishing trend in precipitation, alongside the highly irregular nature of rainfall patterns observed in the region. Additionally, analysis of temperature data reveals a marginal increase; however, the results of the t-test do not suggest any statistically significant trend.


\section{Effect of Climate Variables on Yield of Major Food-crops in Nepal  ― A Time-series Analysis}
\parencite{maharjanEffectClimateVariables2013}
There is a significant increasing trend in the yields of several crops, including paddy, maize, millet, wheat, barley, and potato, over time. 

Rainfall patterns fluctuate annually, showing a lower degree of predictability. While there is an increasing trend in summer rainfall, winter rainfall is on the decline \parencite{maharjanEffectClimateVariables2013}. 

The results of a multivariate regression analysis indicate that the model can explain variations in the yields of food crops, ranging from 40\% for paddy to only 2\% for barley. Although the regression findings show few significant relationships between yield and climate variables, the coefficients can still be used to assess the real effects of these climate variables on food crop yields.

Food crops cultivated in summer are negatively impacted by climate trends, with the exception of paddy, which flourishes in waterlogged conditions. Increased rainfall and higher maximum temperatures adversely affect other summer crops. Conversely, while winter rainfall is decreasing, higher temperatures have positively influenced the yields of winter crops, indicating a differing impact of climate change across seasons. 

Programs aimed at mitigating the effects of climate change on food production should prioritize crops such as maize and potato, as these are facing greater negative impacts compared to other food crops.


\section{Climate Change and its Impact on Nepalese Agriculture}
\parencite{mallaClimateChangeIts2009}

\parencite{mallaClimateChangeIts2009} highlight the multifaceted impacts of climate change on Nepalese agriculture, emphasizing the sensitivity of various crops to altered climatic parameters. For instance, changes in temperature, rainfall, and humidity influence pest and disease dynamics, potentially harming crop yields. Notably, \parencite{mallaClimateChangeIts2009} observed a differential response of maize to temperature increases, with more favorable outcomes in mountainous regions compared to the Terai and hills.

Furthermore, \parencite{mallaClimateChangeIts2009} suggest that rainfed wheat productivity is likely to be more vulnerable in the Terai than in the mid-hills under a climate change scenario. However, they also noted that increased rainfall can mitigate negative impacts on wheat yield across all temperature rise levels.

The study also examines the impact of elevated CO2 and temperature on rice yields. Initially, rice yield increased across the Terai, hills, and mountains. However, with a 4°C temperature increase, yields exhibited a divergent trend: a 3.4\% decrease in the Terai, but continued increases of 17.9\% in the hills and 36.1\% in the mountains, as documented by \parencite{mallaClimateChangeIts2009}. This demonstrates how climate change impacts are not uniform, and vary greatly by region and crop.

\section{Impact of Climate Change on Paddy Production: Evidence from Nepal}
\parencite{pokharauniversitynepalImpactClimateChange2020}
A study conducted in Nepal found a positive correlation between rice yield and rainfall, demonstrating this relationship in both short-term and long-term analyses . Specifically, rainfall was shown to have a causal effect on paddy productivity over the long term. Conversely, the study reported a negative, though statistically insignificant, relationship between both minimum and maximum temperatures and rice yields .

\section{Impact of Climate Change on Wheat Production in Nepal}
\parencite{thapa-parajuliImpactClimateChange2016}
The impact of climate change on wheat production in Nepal has been examined by. \parencite{thapa-parajuliImpactClimateChange2016} study reveals a complex relationship, suggesting that initial temperature increases may boost wheat yields, with an identified optimal minimum of 20°C. However, they caution that this positive impact is limited by a threshold, beyond which further warming could negatively affect production. Furthermore, they also noted a positive correlation between precipitation and wheat yield, highlighting the importance of both temperature and water availability for wheat cultivation in the region."

\section{Overview of agriculture in Nepal: Issues and future strategies}
\parencite{gyawaliOverviewAgricultureNepal2021}

“Climatic diversity, even in a small area of territory, is a unique geographical feature of Nepal ranging from subtropical to arctic in high mountains.

“Since the majority of the country’s population livelihood is dependent on agriculture it is important to understand the factors affecting agricultural output.

Nepal’s long-term agricultural development is at risk due to climate change events, such as erratic rainfall, drought, and floods. These events have a significant impact on crop production, food security, and livelihoods. The country’s agriculture sector is also vulnerable to pests and diseases, which are exacerbated by climate change.

\section{Impact of climate change on agricultural production: A case of  Rasuwa District, Nepal}
\parencite{dawadiImpactClimateChange2022}
“The annual maximum temperature was in an increasing trend (0.01 C/a), however, the annual temperature was in a decreasing trend (0.04 C/a)

\parencite{dawadiImpactClimateChange2022}
“The overall analysis of maize, wheat, and potato production showed a significant increasing trend throughout the study period, but millet production exhibited an insignificant decreasing trend

\parencite{dawadiImpactClimateChange2022}
“Almost half of the respondents perceived that temperature is increasing “About 30.00\% of the respondents did not agree with the temperature change while the rest responded to the temperature decrease

“considerable number of the respondents (62.86\%) were completely unaware of climate change and its impact “Most respondents (62.86\%) had lower than average perceptions of the impact of climate change on agriculture production “However, local people have certainly experienced climate change.

\section{Changes in climate extremes, fresh water availability and vulnerability to food insecurity projected at 1.5°C and 2°C global warming with a higher-resolution global climate model}
\parencite{bettsChangesClimateExtremes2018}
Uncertainties in regional climate and weather extremes are projected to increase with global warming. South Asia is particularly vulnerable to food insecurity due to its high population density and reliance on agriculture.
Vulnerability to food insecurity increases more with global warming.

\section{Impact of Climate Change in the Agricultural System in the Tropical Region of Nepal}
\parencite{bhattaraiImpactClimateChange2021} reported that forced to change cropping pattern due to several factors, Nepalese farmers rely chiefly on rainfall timing, frequency, duration and intensity for growing the crops, which are heavily altered due to climate change.

In the Midwestern Terai region, heavy rainfall has led to a 30\% reduction in crop production.


\section{Impact of Climate Change on Water Resources and Crop Production in Western Nepal: Implications and Adaptation Strategies}
\parencite{risalImpactClimateChange2022}

Similarly, \parencite{risalImpactClimateChange2022} observed a linear but modest increasing trend in temperature for the Banke region using the $NOAA\_RegCM4$ model. However, the analysis revealed less variability in the linear trend, with a decrease in average and maximum temperatures, while minimum temperatures showed an increasing trend.
This decrease in average temperature will cause decline in crop yield leading to temperature stress.

\section{Impact of climate change on biodiversity and food security: a global perspective—a review article}
\parencite{mulunehImpactClimateChange2021}
Globally, CC is expected to reduce cereal production by 1 to 7\% by 2060.
Food security is ensured when everyone has consistent access to sufficient food for an active and healthy life.

\section{Impact of climate change on agricultural production; Issues, challenges, and opportunities in Asia}
\parencite{habib-ur-rahmanImpactClimateChange2022}
Agricultural production is under threat due to climate change in food insecure regions, especially in Asian countries. Various climate-driven extremes, i.e., drought, heat waves, erratic and intense rainfall patterns, storms, floods, and emerging insect pests have adversely affected the livelihood of the farmers. Future climatic predictions showed a significant increase in temperature, and erratic rainfall with higher intensity while variability exists in climatic patterns for climate extremes prediction.

\section{Climate Change: Trends and People’s  Perception in Nepal}
\parencite{devkotaClimateChangeTrends2014}
\parencite{devkotaClimateChangeTrends2014} found that mean temperature rise is slow increasing trend in West Rapti River basin. The devastating floods and incessant rains affect this basin in that: causing extensive damage to standing crops








Assessing the Impact of Climate Variability on Agricultural Productivity and Food Security in Banke, Nepal: Insights from 1990--2020


Climate Adaptation Strategies for Sustainable Agriculture: A Case Study of Janaki Rural Municipality, Banke, Nepal 





























\printbibliography
\end{document}