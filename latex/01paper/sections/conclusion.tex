

\begin{itemize}
  \item	Temperature trends across the Koshi Basin show regional variations. The Himalaya region shows warming in pre-monsoon, while the High Mountain region exhibits significant seasonal temperature changes. The Middle Mountain region shows cooling in monsoon and winter. The Siwalik and Tarai regions experience consistent cooling across seasons, particularly in Tmin and Tavg.
  \item	Seasonal temperatures show heterogeneous trends across decades, temperature categories, and physiographic regions. In overall observed seasonal and annual temperatures trends decreases in starting decades and increase in later decades.
  \item	The Mann-Kendall Test for annual regional trends from 1962 to 2022 revealed varied temperature patterns across the Koshi Basin. In the Himalaya region, Tmax and Tavg slightly increases, while Tmin decreased, though the changes in Tmax and Tavg were not statistically significant. The Middle Mountains exhibited a cooling trend in Tmin and Tavg, with Tmax remaining constant. In the High Mountains, Tmin increased significantly, while Tmax and Tavg experienced notable decrease. The Siwalik region showed consistent declines in all temperature parameters, and the Tarai region also displayed decreasing trends in Tmin, Tmax, and Tavg, though Tmax and Tavg were not statistically significant.
  \item	Temperatures trends in Koshi Basin exhibits heterogeneous patterns across regions, seasons and time periods.

\end{itemize}
