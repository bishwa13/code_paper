

This study analyzed the seasonal and annual trends of maximum temperature (Tmax), minimum temperature (Tmin), and mean temperature (Tavg) across the physiographic regions (Himalaya, High Mountain, Middle Mountain, Siwalik, and Tarai) of the Koshi Basin using observed data from 23 stations between 1962 and 2022. Missing temperature data were filled using the lapse rate method. The Mann-Kendall (M-K) test was employed to assess the consistency of the temperature dataset. The analysis revealed distinct regional and seasonal temperature trends in the Koshi Basin. The seasonal variations were prominent, especially in the Monsoon and Winter. In the Middle Mountain and High Mountain regions, Tmax increased significantly during the Pre-monsoon season, while Tmin decreased significantly during the Monsoon and Winter. Conversely, the Siwalik and Tarai regions experienced more pronounced cooling trends, especially during the Monsoon and Winter. Overall, the Himalaya and High Mountain regions exhibited a cooling trend until the late 1990s, followed by a warming trend. The Middle Mountain region demonstrated similar patterns, with a significant temperature increase after the 1990s. The Siwalik and Tarai regions experienced a general cooling trend, although the Siwalik region exhibited some fluctuations. The significant regional variations in temperature trends were the key findings. Similarly, the Himalaya and High Mountains showed a cooling-warming shift in the late 1990s. All these findings are underscoring the importance of considering both regional and seasonal factors when studying temperature trends in the Koshi Basin.


\textbf{Keywords:} Koshi Basin, temperature trends, spatial variation, physiographic regions, seasonal changes.

