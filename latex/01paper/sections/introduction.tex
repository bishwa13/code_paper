Climate change (CC) contributes to increased greenhouse gas emissions, which will cause a global temperature rise of 1.40 to 5.8°C by the year 2100 compared to 1990 levels \parencite{mccarthy_climate_2001}. The ongoing warming process is expected to significantly affect atmospheric and ecological processes, with wide-reaching consequences for ecosystems and human communities. The threat posed by climate change to mankind is growing, yet many of the most vulnerable individuals are still ignorant of the full effects of global warming \parencite{maharjan_tharu_nodate}. Globally, there has been an increase in the frequency and intensity of severe weather and climate events associated with human-induced climate change since the 1950s. These occurrences are predicted to get worse as long as global warming continues \parencite{intergovernmental_panel_on_climate_change_ipcc_climate_2023} (IPCC, 2021).
The effects of climate change can particularly threaten small, developing nations whose economies and means of subsistence mostly rely on natural resources. One of these nations is Nepal, which is distinguished by its landlocked location, diverse physiographic features within a small region, and difficult hilly terrain \parencite{shrestha_climate_2011}. Nepal, recognized as the fourth most climate-vulnerable nation, exemplifies the challenges posed by climate change \parencite{manandhar_adapting_2011,reilly_climate_in_usa}. Notably, the warming rate in Nepal's Himalayan regions is expected to be greater than the global average, particularly in high-altitude areas \parencite{casestudy_bhattarai,yao_recent_2019,shrestha_maximum_1999}.


Seasonal shifts, such as heavy rainfall during the monsoon season and accelerated glacier melt, contribute to a range of climate-induced hazards, including landslides, floods, and droughts. The effects of these changes are becoming more apparent, and they present a significant challenge to the country’s infrastructure and future development \parencite{pokhrel_climate_2013}. Climate-related disasters are now the primary cause of natural disaster deaths in Nepal, with their frequency increasing in recent years. Because of its sensitivity and lack of preparation, Nepal is regarded as one of the country’s most susceptible to catastrophic weather occurrences \parencite{aksha_spatial_2018}. Environmental changes in the Koshi Basin, one of the significant subbasins of the Ganges River, have led to multiple challenges. Rising temperatures and changing precipitation patterns have increased the frequency and intensity of floods and droughts, leading to increased vulnerability in this region \parencite{bastakoti_agriculture_2017}. The Koshi Basin spans a wide geographical area, ranging from elevations near 100 meters to over 8,000 meters, including the highest point on Earth, Mount Everest. This varied topography results in significant temperature differences within the basin \parencite{bhatt_climate_2014}.

Multiple studies have demonstrated that the Himalayan region is warming at a faster rate than the global average, with temperature trends showing significant variability across altitudes and seasons. While warming rates are generally consistent, they vary depending on altitude, commonly referred to as the elevation dependency of climate warming~\textcite{hingane_longterm_1985,shrestha_observed_2017,Sabin2020,} Their studies revealed an increase in mean annual temperatures, with some areas experiencing a rise of 0.4 degrees Celsius over the past century.~\textcite{shrestha_maximum_1999} investigated maximum temperature trends in Nepal from 1971 to 1994, reporting an average annual temperature increase of 0.06 °C/year. This finding underscores the persistent warming trend in Nepal, particularly in higher-altitude regions. Recent studies further emphasized that the warming rates in the Himalayas are not uniform. Some regions, particularly in the western and central parts of Nepal, have experienced more rapid warming than others. The seasonal patterns also vary, with significant temperature increases noted during winter, followed by spring, autumn, and summer


Multiple studies have demonstrated that the Himalayan region is warming at a faster rate than the global average, with temperature trends showing significant variability across altitudes and seasons.\textcite{hingane_longterm_1985,shrestha_observed_2017, Sabin2020}, Shrestha et al., (2017), Sabin et al. (2020) highlighted that while warming rates are generally consistent, they vary depending on altitude, commonly referred to as the elevation dependency of climate warming. Their studies revealed an increase in mean annual temperatures, with some areas experiencing a rise of 0.4 degrees Celsius over the past century.\textcite{shrestha_maximum_1999} investigated maximum temperature trends in Nepal from 1971 to 1994, reporting an average annual temperature increase of 0.06 °C/year. This finding underscores the persistent warming trend in Nepal, particularly in higher-altitude regions.


Recent studies further emphasize that the warming rates in the Himalayas are not uniform. Some regions, particularly in the western and central parts of Nepal, have experienced more rapid warming than others. The seasonal patterns also vary, with significant temperature increases noted during winter, followed by spring, autumn, and summer \parencite{agarwal_analysis_2016}. In the Koshi Basin, \textcite{shrestha_observed_2017}  found that hill and mountain areas experienced warming between 1975 and 2010, while the plains showed minimal warming or even declines.\textcite{bastakoti_agriculture_2017} observed an increasing variability in minimum temperatures and a narrowing of the range of maximum temperatures. These findings highlight the critical importance of understanding the specific temperature trends in regions like the Koshi Basin to assess their impact on ecosystems and communities. Additionally, studies such as those by\textcite{poudel_spatiotemporal_2020} emphasize the growing complexity of climate impacts, particularly in regions with varying elevations and climatic conditions.


Nepal's climate ranges from subtropical in the Tarai to arctic in the high Himalayas due to its unique physiographic and topographic diversity within a short north-south span. The mean maximum temperature in the Tarai exceeds 30°C, gradually decreasing with altitude to below 22°C in the high mountains. Similarly, the mean minimum temperature ranges from above 18°C in the Tarai to below 6°C in the northwest, reflecting a distinct temperature gradient across the country's varied altitudes \parencite{marahatta_temporal_2009}.


Study of temperature is integral part of climate change studies, those changes can vary based on space and time \parencite{bajracharya_future_2023}, temperature varies largely with altitude \parencite{chand_trend_2019}. Research in Nepal demonstrates the urgency of addressing climate change, as the country faces heightened vulnerability to extreme weather events and significant challenges in preparedness \parencite{chapagain_unpacking_2021}. This research aims to contribute to the growing body of knowledge on climate trends in the Himalayas and their implications for environmental and social resilience in vulnerable regions like the Koshi Basin.

