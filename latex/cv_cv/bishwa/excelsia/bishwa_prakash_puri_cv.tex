\documentclass[a4paper, 09pt]{extarticle}
\usepackage[utf8]{inputenc} 
\usepackage{geometry} 
\geometry{letterpaper, margin=0.2in} 
\usepackage{titlesec} % For section title formatting
\usepackage{enumitem} 
\usepackage{hyperref} 
\usepackage{tabularx}
\usepackage{fancyhdr}

\setlist{noitemsep} 
\titleformat{\section}{\large\bfseries}{\thesection}{2em}{}[\titlerule] 
\titlespacing*{\section}{1pt}{\baselineskip}{\baselineskip}
\begin{document}



\begin{center}
\textbf{\huge Bishwa Prakash Puri}\\[4pt] 
Assistant Lecturer, College of Applied Sciences-Nepal, Kathmandu, Nepal\\
\href{mailto:mabishwapuri@gmail.com}{mabishwapuri@gmail.com}, 
\href{tel:+9779843314630}{+977 9843314630} \\
\href{https://workbishwa.github.io/imBishwa/}{workbishwa.github.io/imBishwa/}

\end{center}

Growing up in Nepal's Koshi Basin, I developed a deep commitment to understanding and addressing environmental challenges in Himalayan communities. My expertise spans climate data analysis, multi-hazard risk assessment, and hydrological modeling, supported by strong programming skills in Python, R, and GIS technologies. As an environmental consultant and assistant lecturer in Environmental Modeling, I combine technical proficiency with practical field experience to develop solutions for disaster risk reduction and sustainable development. My research on temperature trends and agricultural impacts in the Koshi Basin, along with my experience in environmental impact assessments, has equipped me to bridge scientific modeling with community-focused solutions.

\section*{EDUCATION}

\textbf{Tribhuvan University (MSc.)} \hfill \textbf{2024}\\[2pt] 
College of Applied Sciences-Nepal, Kathmandu, Nepal\\
Master of Science in Environmental Science

\vspace{5pt}

\noindent
\textbf{Tribhuvan University (BSc.)} \hfill \textbf{2019}\\[2pt] 
Tri-Chandra Multiple Campus, Kathmandu, Nepal\\
Bachelor of Science in Environmental Science

\vspace{5pt}

\section*{ACADEMIC EXPERIENCE}
\noindent
\textbf{College of Applied Sciences-Nepal} \hfill Kathmandu, Nepal\\[2pt] 
\textit{Assistant Lecturer} \hfill November 2024--January 2025 
\begin{itemize}
    \item Teaching Environmental Modeling to MSc 4th-semester students, focusing on hydrological modeling (HEC-HMS, HEC-RAS), climate modeling (SDSM), and geospatial analysis (GIS).
    \item Designing and delivering lectures, leading hands-on practical sessions, and mentoring students in applying modeling tools to environmental challenges.
    \item Serving as a thesis co-supervisor, guiding research projects and ensuring academic and professional excellence.
\end{itemize}

\vspace{5pt}

\section*{PROFESSIONAL EXPERIENCE}
\noindent
\textbf{Freelancer} \hfill Nepal\\[2pt] 
\textit{Environmental Consultant, Researcher, Business Development Consultant} \hfill  August 2024--Present
\begin{itemize}
    \item Preparation of EIA, IEE, and Monitoring Reports for hydropower projects.
    \item Conduct fieldwork and assess environmental impacts of hydroelectric projects.
    \item Consultations with stakeholders.
    \item Preparation and presentations of reports.
    \item Business development and growth strategies for small businesses consultancy firms.
\end{itemize}

\noindent
\textbf{Quest Forum} \hfill Kathmandu, Nepal\\[2pt] 
\textit{Environmental Assistant} \hfill January 2023--June 2023
\begin{itemize}
    \item Assisted in environmental studuies including preparation of reports, preparation of Terms of References (TOR), faciliation of public hearing, data collection, preparation of presentations.
    \item Compilation of technical and financial documents for Biding.
    \item Assisted in field visits for environmental monitoring in various industries from different sectors such as Brick, Cement, Beverage, and Pharmaceuticals.
    \item Assisted in Ppreparation of Environmental Monitoring Report.
    \item Communicated findings and recommendations to clients, stakeholders, and regulatory bodies.
\end{itemize}

\noindent
\textbf{Nepal Environmental and Scientific Services (NESS)} \hfill Kathmandu, Nepal\\[2pt] 
\textit{MSDS Officer Intern} \hfill July 2022 -- December 2022
\begin{itemize}
    \item Prepared laboratory consumables inventory and created material safety data sheets (MSDS).
\end{itemize}

\section*{PUBLICATIONS}
\noindent
\textbf{Journal Articles}
\begin{itemize}
    \item Puri, B. P., Adhikari, T. R., et al. \href{https://doi.org/10.3126/jhm.v12i1.72654}{\textit{Spatial and Temporal Variations of Surface Air Temperature (1962–2022) across Physiographic Regions in the Koshi Basin, Nepal.}} Published in \href{https://soham.org.np/}{\textit{SOHAM.}}
    \item Gautam, R., Puri, B. P, et al. \href{https://oaskpublishers.com/assets/article-pdf/impact-of-climate-variability-on-agricultural-productivity-and-food-security--in-banke-nepal-insights-from-1990-2020.pdf}{\textit{Impact of Climate Variability on Agricultural Productivity and Food Security in Banke, Nepal: Insights from 1990--2020}}. Published in \href{https://oaskpublishers.com/}{\textit{OASK Publishers.}}
    \item Research Paper (In Progress): \textit{Assessing Soil Nutrient Dynamics and Their Impacts on Crop Yields: A Case Study of Janaki Rural Municipality, Banke, Nepal}. 
\end{itemize}

\noindent
\textbf{Thesis}
\begin{itemize}
    \item Temperature Trend Analysis Along with Paddy Production Trend and Their Impact on Paddy Cultivation in Koshi Basin, Nepal.

\end{itemize}

\section*{RESEARCH AND FIELD EXPERIENCE}
\begin{itemize}
    \item Conducted flood frequency analysis using Gumbel’s Distribution Method and HEC-HMS, and developed river channel geometry for Balkhu River Basin using HEC-RAS.
    \item Simulated rainfall data using Statistical Downscaling Modelling (SDSM) during the Flood and Hydrological Modeling Project in Surkhet Weather Station, Nepal.
    \item Conducted geological data collection, input, and analysis in Kakani Ranipauwa Area, Nuwakot District, Nepal.
    \item Geological and Ecological Field Study, Malekhu – Pokhara, Nepal: Conducted geo-environmental studies on rock mass classification, slope stability, landslides, and biodiversity; created environmental geological maps and assessed fluvial deposits.
    \item Led community-based solid waste segregation initiatives at Sunrise English Boarding School, Sainbu, Lalitpur, Nepal, including managing solid waste and conducting field-based data collection.
    \item Emission inventory from cement production in Nepal:  Annual pollutant emission that generate from the Shivam and Hongshi cement industries was calculated and estimated by the help of default emission factor for different pollutant.
    \item Emission inventory from hospital: Conducted an emission inventory analysis of NOx, CO, and SOx from Bir Hospital and Grande Hospital, including calculations of mercury and lead emissions during clinical waste incineration.
    \item Water quality monitoring of ground water and irrigation water: Conducted water quality monitoring of surface and groundwater of Koplu Khola, Nuwakot, analyzing parameters such as temperature, pH, EC, TDS, TSS, heavy metals, and coliform bacteria.
    \item Measurement and comparison of noise level (equivalent, percentile) in residential, public and workplaces: Compared and identified the noise level within Balkhu and nearby areas.
\end{itemize}

\vspace{5pt}
\section*{SKILLS}
\begin{itemize}
    \item \textbf{Programming:} Python, R
    \item \textbf{Software:} ArcGIS, HEC-HMS, HEC-RAS, SDSM, LaTeX, Markdown, Git
    \item \textbf{Technical:} Field Surveys, EIA/IEE/BES, Report Writing, Stakeholder Consultations
\end{itemize}

\section*{TRAINING AND CERTIFICATIONS}
\begin{itemize}
    \item Geographic Information System (GIS) Certification – 1 Month (2020)
    \item Chemical Management and MSDS Certification – 2 Weeks (2022)
    \item Environmental Impact Assessment Certification – 3 Weeks (2020)
\end{itemize}

\section*{Language Skills}

\noindent
\begin{tabularx}{\textwidth}{|l|X|X|X|}
\hline
\textbf{Language} & \textbf{Reading} & \textbf{Writing} & \textbf{Speaking} \\ \hline
Nepali & Native & Native & Native \\ \hline
English & Excellent & Excellent & Good \\ \hline
Hindi & Excellent & Beginners & Intermediate \\ \hline
\end{tabularx}

\section*{REFERENCES}

\noindent
\textbf{Dr. Bhupendra Devkota}\\[2pt]
\textit{Principal, Professor, College of Applied Sciences-Nepal}\\
\href{mailto:bhupendra.devkota@gmail.com}{bhupendra.devkota@gmail.com}

\vspace{10pt}

\noindent
\textbf{Dr. Tirtha Raj Adhikari}\\[2pt]
\textit{Professor Emeritus, Tribhuvan University}\\
\href{mailto:tirtha43@gmail.com}{tirtha43@gmail.com}

\vspace{5pt}

\noindent
\textbf{Surendra Dev Bhatta}\\[2pt]
\textit{Executive Director, Green Globe International Pvt. Ltd.}\\
\href{mailto:surenbtta@gmail.com}{surenbtta@gmail.com}

\end{document}
