\documentclass[12pt]{article}
\usepackage[margin=0.5in]{geometry}
\usepackage{amsmath}
\usepackage{setspace}
\usepackage{hyperref}

\title{\textbf{Statement of Research Interest}}
\author{Bishwa Prakash Puri}
\date{}

\begin{document}

\maketitle
\doublespacing

I am deeply motivated to pursue research that integrates environmental science with advanced data-driven methodologies such as remote sensing, artificial intelligence, and geospatial analytics. My academic background in Environmental Science (B.Sc. and M.Sc., Tribhuvan University), coupled with professional and academic experience in hydrological modeling, environmental impact assessment, and climate variability analysis, has laid a strong foundation for interdisciplinary research.

My recent work as an Assistant Lecturer in Environmental Modeling (Hydrology) involved teaching advanced modeling tools such as HEC-HMS, HEC-RAS, and SDSM, where I guided graduate students in applying computational methods to hydrological challenges. My ongoing research and publications focus on climate change impacts, soil nutrient dynamics, and agricultural productivity in Nepal, emphasizing the value of spatiotemporal data analysis in addressing real-world environmental problems.

I possess soft programming skills, particularly Python and R, with practical experience in applying libraries such as \texttt{scikit-learn}, \texttt{GDAL}, \texttt{rasterio}, and others for environmental data analysis and spatial data visualization, along with \LaTeX{} for report preparation. I am familiar with google earth engine, leaflet, api integration, django and git for colaboration.  Furthermore, my hands-on experience with emission inventory, environmental monitoring, and data-intensive field research complements the vision of Dr. Liu’s lab. I am particularly drawn to projects involving the application of AI and remote sensing to analyze atmospheric phenomena and extreme weather events. I believe that by merging my practical expertise with cutting-edge computational techniques, I can meaningfully contribute to the lab’s efforts to advance environmental resilience.

Joining your research team would allow me to build upon my experience and develop transformative solutions to pressing environmental challenges through data science.

\vspace{1cm}

\noindent
Sincerely, \\
\textbf{Bishwa Prakash Puri}

\end{document}
