
\subsection{Research clusters}
The proposed research on sustainable agriculture and climate change adaptation in the Murray-Darling Basin aligns with a diverse array of research clusters, reflecting its interdisciplinary scope across STEM and related fields. Within the core STEM disciplines, it encompasses Science, leveraging environmental science to study climate impacts on agriculture; Technology, utilizing tools like Remote Sensing and Artificial Intelligence (AI); Engineering, applying data-driven frameworks to real-world agricultural solutions; and Maths, relying on statistical modeling and data analytics for predictive insights. It strongly integrates STEM (Education), designing educational interventions to empower farmers with scientific knowledge. The project directly engages with Agriculture, focusing on sustainable practices to enhance resilience, and employs Data Analytics and Big Data to process global, regional, and local datasets. It falls under Sustainable/Environmental, addressing climate adaptation and resource conservation, and uses Spatial Reasoning through GIS to map vulnerabilities. Biology is key in examining plant-soil interactions, while Computing underpins the use of Python and GIS software. Artificial Intelligence drives machine learning models, and Information Systems supports the dissemination of insights via digital platforms. The research also incorporates Design Think, crafting user-focused educational tools, and Enviro/Geo, drawing on environmental geology and geographic analysis. Beyond the provided list, additional nominated categories include Climate Science, for its focus on climate change dynamics; Community Engagement, for empowering agricultural communities; Policy Development, for offering actionable insights to policymakers; Soil Science, for studying soil health under climate stress; and Water Management, addressing the Basin’s critical water scarcity issues. Together, these 21 clusters highlight the project’s comprehensive approach, blending cutting-edge science, technology, and education to tackle a pressing global challenge in a climate-vulnerable region.




