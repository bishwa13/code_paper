


\subsection{Research Project Objectives}
\begin{enumerate}
    \item Develop Integrated Sustainable Agriculture Practices: Investigate the application of advanced methodologies—such as Remote Sensing, Geographic Information Systems (GIS), and Artificial Intelligence (AI)—to create innovative and sustainable agricultural practices that enhance resilience to climate variability in the Murray-Darling Basin. This includes assessing climate variable impacts on plant growth and soil health.
    
    \item Enhance Information for agricultural Community/Business Empowerment in the Murray-Darling Basin: Design and evaluate education-driven interventions aimed at equipping local farmers and stakeholders with digital tools and scientific knowledge, fostering community-based adaptation strategies and bridging the gap between research and practice.
    
    \item Establish a Data-Driven Framework for Climate Change Adaptation: Develop a comprehensive, interdisciplinary model that integrates global, regional, and local data analytics with practical educational initiatives to predict climate impacts on agricultural farmland, recommending adaptive solutions for improved crop productivity and long-term sustainability.
\end{enumerate}

\subsection{Research Thesis Vision}

My passion for sustainable agriculture and climate change  adaptation has been deeply shaped by my upbringing in Banke, Nepal, a region where the rhythm of life is intrinsically linked to agricultural practices. Growing up amidst vast croplands, I have witnessed firsthand the evolving patterns of cultivation, the fluctuating fortunes of agricultural productivity, and the increasing encroachment upon arable land. However, it was the profound impact of climatic disasters that truly ignited a lifelong passion to understand and address the challenges facing Nepal's agricultural sector. I vividly recall the devastating floods that inundated Banke and surrounding areas during my childhood, a stark reminder of the vulnerability of our agricultural systems to extreme weather events. Subsequent occurrences of inundation, droughts, and erratic monsoons have further underscored the fragility of crop productivity in the face of a changing climate.Agriculture forms the backbone of Nepal's economy, engaging a significant majority of the population and contributing substantially to the nation's GDP \parencite{krupnikAgronomicSocioeconomicEnvironmental2021}. However, this sector is increasingly threatened by the intensifying impacts of climate change, which manifest as rising temperatures, altered precipitation patterns, and a greater frequency of extreme weather events like floods and droughts \parencite{dawadiImpactClimateChange2022}. These changes not only impact crop yields and livestock health but also pose significant threats to food security and the livelihoods of countless families who depend on agriculture \parencite{risalImpactClimateChange2022}. Understanding climate change offers a path towards sustainable agriculture, which is a promising pathway to mitigate these vulnerabilities by promoting practices that enhance soil health, conserve water resources, reduce reliance on harmful chemicals, and ultimately build more resilient farming systems \parencite{factors_dahal_2021}. This journey led me to my Master’s in Environmental Science, where my thesis, "Impact of Climatic Variables on Crop Production and the Role of Environmental Factors in Janaki Rural Municipality, Banke, Nepal," used statistical modeling to uncover how climate, soil nutrients, and farming practices intertwine. These experiences now guide my focus on the Murray-Darling Basin, a vital agricultural region grappling with drought and water scarcity. 

Agriculture underpins the Murray Darling Basin’s economy, producing 40\% of Australia’s food, yet climate change rising temperatures, erratic rainfall, and extreme events threatens yields and sustainability. All the journey has provided me with a solid foundation in understanding these complex relationships. Through statistical modeling, I analyzed the intricate connections between crop yields and climatic factors in my locality, also examining the influence of pests, soil nutrients, fertilizer inputs, irrigation, and land-use change. This research provided valuable insights into how climatic variability disrupts traditional farming methods. Furthermore, my involvement in research projects related to environmental geology, disaster risk management, and water quality assessments has honed my analytical skills and provided a multidisciplinary perspective essential for tackling complex environmental challenges. My familiarity in programming languages such as Python and LaTeX, statistical tools like SPSS and Excel, GIS platforms including QGIS and ArcGIS, and diverse data analysis techniques provides me with the technical foundation required to conduct robust research in this domain. Education is central—STEM tools will empower farmers, bridging science and practice for a sustainable future. 


Education is central to this vision. Research shows it empowers farmers to adopt sustainable methods and build adaptive capacity \parencite{prettyIntensificationRedesignedSustainable2018}. In the Basin, where 2.4 million rely on agriculture, I aim to develop STEM-based tools—workshops and information sharing standards—to distill scientific insights for farmers and leaders. This bridges research and practice, fostering climate literacy and resilience. My diverse expertise in climate change, agriculture, water-soil chemistry, and disaster vulnerabilities, paired with robust technical skills, underpins my ability to achieve this objective.

This research aligns with the United Nations Sustainable Development Goals, particularly SDGs 2 (Zero Hunger), 4 (Quality Education), and 13 (Climate Action). It advances the global discourse on sustainable agriculture by providing practical solutions for the Murray-Darling Basin, a region emblematic of climate-vulnerable areas. By integrating advanced scientific methods with education, the project ensures that knowledge directly benefits local communities.

\subsection{Abstract}

This research investigates climate change impacts on agriculture in the Murray-Darling Basin, Australia’s agricultural heartland, producing 40\% of the nation’s food, yet increasingly vulnerable to rising temperatures, erratic rainfall, and water scarcity \parencite{postResearchInformingPolicy2025}. Drawing from my upbringing in Nepal’s Terai region, where floods and droughts disrupted farming, and my Master’s thesis on climatic variability’s effects on crop production, I propose a multidisciplinary approach to enhance resilience. Employing advanced tools—Remote Sensing, GIS, and Artificial Intelligence—the study will develop predictive models to assess climate effects on crop yields and soil health, testing sustainable practices like agroforestry and precision irrigation.

Central to this project is a dual focus on technology and education. Quantitative methods, including spatial analysis and data analytics, will create a robust framework integrating global, regional, and local data to recommend adaptive solutions for sustainable productivity. Simultaneously, STEM-based educational interventions—workshops and digital tools—will empower the Basin’s 2.4 million residents, fostering community-driven adaptation and bridging research with practice. Research underscores education’s role in enhancing farmers’ adaptive capacity and climate literacy \parencite{prettyIntensificationRedesignedSustainable2018}, a principle I aim to apply here.

Using a mixed-methods approach, the study will combine climate and agricultural data analysis with field surveys, evaluating both technological efficacy and educational impact. My expertise in Python, LaTeX, SPSS, Excel, QGIS, ArcGIS, and data analysis, honed through environmental science and disaster management projects, equips me to tackle these challenges. Outcomes include predictive tools, validated practices, and an educational model, aligning with SDGs 2 (Zero Hunger), 4 (Quality Education), and 13 (Climate Action). By offering actionable solutions for the Murray-Darling Basin—a microcosm of climate-vulnerable regions—this research advances global sustainable agriculture discourse, ensuring knowledge directly benefits communities.

\subsection*{Additional Information}

\textbf{Skills}

My multidisciplinary skill set equips me to excel in this research endeavor, spanning academic, industry, education, and additional domains:

\begin{itemize}
    \item \textbf{Academic:} My Master’s in Environmental Science provided a robust foundation in climate change, agriculture, and water-soil chemistry, culminating in my thesis, "Impact of Climatic Variables on Crop Production and the Role of Environmental Factors in Janaki Rural Municipality, Banke, Nepal." I have conducted statistical modeling to analyze climate-crop relationships and participated in research projects on environmental geology, disaster risk management, and water quality assessments, honing my analytical and interdisciplinary research capabilities.
    
    \item \textbf{Industry:} Proficiency in industry-standard tools enhances my practical research application. I am skilled in programming languages (Python, LaTeX), statistical software (SPSS, Excel), and GIS platforms (QGIS, ArcGIS), enabling me to process large datasets, develop predictive models, and visualize spatial data effectively. These skills align with industry demands for data-driven solutions in agriculture and environmental management.
    
    \item \textbf{Education:} I have a strong commitment to education, evidenced by my vision to design STEM-based interventions—workshops and digital tools—to empower Murray-Darling Basin farmers. My experience in analyzing complex scientific data and translating it into actionable insights equips me to develop educational frameworks that bridge research and practice, fostering climate literacy and sustainable practices among stakeholders.
    
    \item \textbf{Other:} My diverse expertise includes disaster vulnerability assessment, critical for understanding climate risks in agricultural regions, and multilingual communication (Nepali, Hindi, English), facilitating collaboration with diverse communities. Additionally, my problem-solving skills, honed through design thinking and interdisciplinary projects, support innovative approaches to sustainable agriculture challenges.
\end{itemize}

These skills collectively underpin my ability to conduct rigorous research, integrate advanced technologies, and deliver impactful educational outcomes, aligning with the project’s goals and the United Nations Sustainable Development Goals (SDGs 2, 4, 13).